\documentclass[12pt, a4paper,titlepage]{report}
\usepackage[a4paper,left=2cm,right=2cm,top=2.5cm,bottom=2.5cm]{geometry}
\usepackage{
  mathptmx, % use ~Times New Roman
  wrapfig,
  graphicx, 
  titlesec, 
  fancyhdr,
  tikz, 
  pgfplots, 
  amsmath,
  amssymb,
  subcaption, 
  amsthm,
  multirow,
  algorithm, 
  algpseudocode,
  listings, 
  newtxtt, 
  longtable, 
  booktabs
}

\usepackage[]{hyperref}
\hypersetup{
  colorlinks=false,
  urlbordercolor=black,% url borders will be red
  pdfborderstyle={/S/U/W 1}% border style will be underline of width 1pt
}


\lstset{ % General setup for the package
	basicstyle=\small\sffamily,
	numbers=none,
 	numberstyle=\tiny,
	frame=tb,
	tabsize=4,
	columns=fixed,
	showstringspaces=false,
	showtabs=false,
	keepspaces,
	commentstyle=\color{red},
	keywordstyle=\color{blue}
}


\renewcommand{\abstractname}{\large Abstract}
\newcommand{\passthrough}[1]{#1}
\lstset{basicstyle=\ttfamily\footnotesize,breaklines=true}
\setlength{\headheight}{15pt}
\DeclareMathOperator*{\opt}{opt}
\pgfplotsset{compat=1.17}
\newcommand{\pgfenv}{
\pgfplotsset{
        standard/.style={%Axis format configuration
        axis x line=middle,
        axis y line=middle,
        enlarge x limits=0.15,
        enlarge y limits=0.15,
        every axis x label/.style={at={(current axis.right of origin)},anchor=north west},
        every axis y label/.style={at={(current axis.above origin)},anchor=north east},
        every axis plot post/.style={mark options={fill=white}}
        }
    }
  }

\setcounter{tocdepth}{4}
\setcounter{secnumdepth}{4}
% Revert to Computer Modern Symbol math alphabet
\DeclareMathAlphabet{\mathcal}{OMS}{cmsy}{m}{n}
\SetMathAlphabet{\mathcal}{bold}{OMS}{cmsy}{b}{n}

% Chapter Titling: Chapter [0-9] LEFT, Chapter Title RIGHT
\newcommand*{\justifyheading}{\raggedleft}
\titleformat{\chapter}[display]
{\normalfont\large}{\MakeUppercase\chaptertitlename \ \ \thechapter}
{20pt}{\Huge\bfseries\justifyheading}


\begin{document}

% Forewords + TOC Page header Style
% pageNumber -- Chapter Title -------- | ------- Chapter Title -- pageNumber
\pagestyle{fancy}
\renewcommand{\headrulewidth}{0pt} % to remove line on header
\renewcommand{\footrulewidth}{0pt} % to remove line on footer
\renewcommand{\chaptermark}[1]{\markboth{#1}{}}
\fancyhead[LE]{\thepage \ \ }
\fancyhead[RO]{\MakeUppercase\leftmark \ \ \thepage}
\fancyfoot[C] {\thepage}


\begin{abstract}
This document is a formal specification of the Alef programming language.
It specifies the syntactic features of the language and explains it semantics. 
Any compiler or interpreter for this language must conform to these specifics.
Grammar rules are expressed using EBNF: 
\begin{lstlisting}
Production  = production_name "=" [ Expression ] "." .
Expression  = Alternative { "|" Alternative } .
Alternative = Term { Term } .
Term        = production_name | token [ "..." token ] | Group | Option | Repetition .
Group       = "(" Expression ")" .
Option      = "[" Expression "]" .
Repetition  = "{" Expression "}" .   (* Repeating 0 times is allowed. *)
\end{lstlisting}
(Taken from \url{https://golang.org/ref/spec}.)

\vspace*{1em}
\noindent
Notes about the current implementation and examples are in italic: 

{\it This is an example.}

\noindent
Code snippets and grammar rules are represented using listings:
\begin{lstlisting}
    this is a code snippet.
\end{lstlisting}

\vspace*{1em}
\noindent
This document is a draft. 
\end{abstract}

\tableofcontents    


% Corpus Header Style
% pageNumber -- ChapterTitle ----- Chapter | Chapter ------ Section -- pageNumber
\pagestyle{fancy}
\renewcommand{\headrulewidth}{0pt}
\renewcommand{\chaptermark}[1]{\markboth{#1}{}}
\fancyhf{}
\fancyhead[LE]{\thepage \ \ \MakeUppercase\leftmark}
\fancyhead[RE, LO]{\MakeUppercase\chaptertitlename \ \ \thechapter}
\fancyhead[RO]{\rightmark \ \ \thepage}
\fancyfoot[C]{\thepage}

% TEX root = ../main.tex
\chapter{Execution model}

Alef is a concurrent programming language designed for
systems software. Exception handling, process management, and synchronization
primitives are implemented by the language. Expressions use the same syntax
as C, but the type system is substantially different. Alef supports object-oriented
programming through static inheritance and information hiding. The language does not
provide garbage collection, so programs are expected to manage their own memory.
This manual provides a description of the syntax and semantics of the
implementation.

\noindent
In the Alef execution model the term \textit{process} refers to a preemptively 
scheduled thread of execution. A process may contain severl \textit{tasks}, which 
are non-preemptively scheduled coroutines within a process. 

The memory model does not define the sharing of memory between processes; 
on a shared memory computer, processes will typically share the same address 
space, while on a multicomputer processes may be located on physically distant 
nodes with access only to local memory. In such a system, processes would not 
share the same address space, therefore must communicate using message passing. 
A group of tasks executing within the context of a process are defined to be 
in the same address space. Tasks are scheduled during communication (sending or 
receiving on/from channels) and synchronization (lock, unlock) operations. 
The term \textit{thread} is used wherever the distinction between a process and 
a task is unimportant. 

\textit{
In the current implementation, processes are implemented as plain OS forks, 
while tasks are implemented as green threads, using a round
robin scheduler implemented in the runtime -- scheduling of tasks is realized 
through simple calls to \texttt{\_alef\_yield}.}
 
% !TEX root = ../main.tex

\hypertarget{type-system}{%
\chapter{Type system}\label{type-system}}

\hypertarget{basic-types}{%
\section{Basic types}\label{basic-types}}

Alef defines a small set of basic types:

\begin{table}[h]
  \centering
\begin{tabular}{|c|c|c|}
\hline
name  & size    & type                  \\ \hline
byte  & 8 bits  & unsigned byte         \\ \hline
sint  & 16 bits & signed short int      \\ \hline
usint & 16 bits & unsigned short int    \\ \hline
int   & 32 bits & signed integer        \\ \hline
uint  & 32 bits & unsigned integer      \\ \hline
float & 64 bits & floating point        \\ \hline
lint  & 64 bits & long signed integer   \\ \hline
ulint & 64 bits & long unsigned integer \\ \hline
chan  & 32 bits & channel               \\ \hline
poly  & 64 bits & polymorphic type      \\ \hline
\end{tabular}
\end{table}

where the given size is the minimum number of bits required to represent
that type. The \passthrough{\lstinline!float!} type should be the
highest precision floating point provided by the hardware, therefore
format and precision are implementation dependent. The alignment of
these types is implementation dependent as well. The
\passthrough{\lstinline!chan!} type is actually a pointer to a
runtime-system-defined object, and must be allocated before use, and are
the size of a pointer. Polymorphic types are represented by a pointer
and an hash of the type it currently represents. For a given
implementation the polymorphic type has the same size as the following
aggregate definition:

\begin{lstlisting}
aggr Polytype 
{
  void* ptr;
  int   hash;
};
\end{lstlisting}

The void type performs the special task of declaring procedures
returning no value and as part of a derived type to form generic
pointers. The void type may not be used as a basic type. The integral
types are \passthrough{\lstinline!int!}, \passthrough{\lstinline!uint!}
, \passthrough{\lstinline!sint!}, \passthrough{\lstinline!usint!} ,
\passthrough{\lstinline!byte!}, \passthrough{\lstinline!lint!} and
\passthrough{\lstinline!ulint!} . The arithmetic types are the integral
types and the type \passthrough{\lstinline!float!}. The pointer type is
a type derived from the \passthrough{\lstinline!\&!} (address of)
operator or derived from a pointer declaration.

From the point of view of the lexer types are all identifiers; however, the compiler 
needs to distinguish identifiers and typenames; therefore, for the grammars, 
identifiers naming types will be tokens of type \passthrough{\lstinline!Typename!}.
\begin{lstlisting}
BaseType = Typename [ ( ChanSpec | GenericInstantiation ) ] .
GenericInstantiation = "[" Variant "]" . 
Variant = TypeCast { "," TypeCast } . 
ChanSpec = "(" Variant ")" [ ChanBufDim ] .  
ChanBufDim = "[" Expression "]" . 
\end{lstlisting}

\hypertarget{channel-types}{%
\subsection{Channel types}\label{channel-types}}

The type specified by a chan declaration is actually a pointer to an
internal object with an anonymous type specifier. Because of their
anonymity, objects of this special type cannot be defined in
declarations; instead they must be created by an alloc statement
referring to a chan.

\hypertarget{sync-channels}{%
\subsubsection{Sync channels}\label{sync-channels}}

A channel declaration without a buffer specification produces a
synchronous communication channel. Threads sending values on the channel
will block until some other thread receives from the channel. The two
threads rendezvous and a value is passed between sender and receiver.

\hypertarget{async-channels}{%
\subsubsection{Async channels}\label{async-channels}}

If buffers are specified, then an asynchronous channel is produced. A
send operation will complete immediately while buffers are available,
and will block if all buffers are in use. A receive operation will block
if no value is buffered. If a value is buffered, the receive will
complete and make the buffer available for a new send operation. Any
senders waiting for buffers will then be allowed to continue.

\hypertarget{variant-channels}{%
\subsubsection{Variant channels}\label{variant-channels}}

If multiple types are specified in a channel definition, the channel
supplies a variant protocol. A variant protocol allows messages to be
demultiplexed by type during a receive operation. A form of the alt
statement allows the control flow to be modified based on the type of a
value received from a channel supplying a variant protocol.

\hypertarget{polymorphic-types}{%
\subsection{Polymorphic types}\label{polymorphic-types}}

The polymorphic type can be used to dynamically represent a value of any
type. A polymorphic type is identified by an identifier defined in a
polymorphic type definition or as a parameter to a polymorphic abstract
data type. Distinct identifiers represent a value of the same structure
but are different for the purposes of type checking. A polymorphic value
is represented by a fat pointer, which consists of an integer tag (an
identifier of the type, such as an hash) and a pointer to a value. Like
channels, storage for the data must be allocated by the runtime.

\hypertarget{enumerations}{%
\subsection{Enumerations}\label{enumerations}}

Enumerations are types whose value is limited to a set of integer
constants. These constants, the members of the enumeration, are called
enumerators. Enumeration variables are equivalent to integer variables.
Enumerators may appear wherever an integer constant is legal. If the
values of the enumerators are not defined explicitly, the compiler
assigns incrementing values starting from 0. If a value is given to an
enumerator, values are assigned to the following enumerators by
incrementing the value for each successive member until the next
assigned value is reached.

\hypertarget{derived-types}{%
\section{Derived types}\label{derived-types}}

Types are derived in the same way as in C. Operators applied in
declarations use one of the basic types to derive a new type. The
deriving operators are:

\begin{lstlisting}
*     create a pointer to 
&     yield the address of
()    a function returning 
[]    an array of
\end{lstlisting}

These operators bind to the name of each identifier in a declaration or
definition. Both the \passthrough{\lstinline!\&!} prefix operator and
the \passthrough{\lstinline!()!} operator have distinct rules in prefix
unary expressions and basic declarations, respectively.

\begin{lstlisting}
ArraySpec = "[" [ Expression ] "]" { "[" [ Expression ] "]" } .
PtrSpec = "*" { "*" } . 
\end{lstlisting}

\hypertarget{array-specifiers}{%
\subsection{Array Specifiers}\label{array-specifiers}}

The dimension of an array must be non-zero positive constant. Arrays
have a lower bound of $0$ and an upper bound of $n-1$, where $n$ is the value
of the constant expression.

\hypertarget{complex-types}{%
\section{Complex types}\label{complex-types}}

Complex types may be either aggregates, unions, tuples, or abstract data
types. These complex types contain sequences of basic types, derived
types and other complex types, called members. Members are referenced by
tag or by type, and members without tags are called unnamed. Arithmetic
types, channel types, tuples, and complex types may be unnamed. Derived
types may not be left unnamed. Complex unnamed members are referenced by
type or by implicit promotion during assignment or when supplied as
function arguments. Other unnamed members allocate storage but may not
be referenced. Complex types are compared by structural rather than name
equivalence.

An aggregate is a simple collection of basic, derived\footnote{\textbf{@TODO}:
  Can a function be defined inside an aggregate?} and complex types with
unique storage for each member. Unions, instead, store each member in
the same storage and its size is determined by the size of the largest
member. Abstract data types, adt's, are comparable to aggregates, but
also has a set of functions to manipulate objects of its type and a set
of visibility attributes for his members, to allow information hiding.

The declaration of complex types bind identifiers to such types, and
after declaration such identifier can be used wherever a basic type is
permitted. New type bindings may be defined from derived and basic types
using the \passthrough{\lstinline!typedef!} statement.

The complex types are \passthrough{\lstinline!aggr!},
\passthrough{\lstinline!adt!}, \passthrough{\lstinline!union!} and
\passthrough{\lstinline!tuple!}\footnote{\textbf{@TODO}: Maybe we could
  clarify, in a ``mechanical'' way, how complex type fit in the type
  system?}\footnote{\textbf{@TODO}: Clarify why tuples are different
  from other complex types, i.e.~why they can be declared as basic types
  can (inline).}.

\begin{lstlisting}
Type = BaseType | [ "tuple" ] "(" TupleList ")" .
TupleList = TypeCast "," TypeCast { "," TypeCast } .
\end{lstlisting}

What follows is a in-depth analysis of the meaning of the complex types.
Grammar rules for deriving such constructs will be given in the
Declarations section.

\hypertarget{tuples}{%
\subsection{Tuples}\label{tuples}}

A tuple is a collection of types forming a single object which can be
used in the place of an unnamed complex type. The individual members of
a tuple can only be accessed by assignment. When the declaration of a
tuple would be ambiguous because of the parenthesis (for instance in the
declaration of an automatic variable), use the keyword tuple:

\begin{lstlisting}
void
f()
{
    int a;
    tuple (int, byte, Rectangle) b;
      int c;
}
\end{lstlisting}

Type checking of tuple expressions is performed by matching the shape of
each of the component types. Tuples may only be addressed by assignment
into other complex types or l-valued tuple expressions. A parenthesized
list of expressions forms a tuple constructor, while a list of l-valued
expressions on the left hand side forms a destructor.

For example, to make a function return multiple values:

\begin{lstlisting}
(int, byte*, byte) func()
{
      return (10, "hello", 'c');
}
void main()
{
      int a;
      byte* str;
      byte c;
      (a, str, c) = func();
}
\end{lstlisting}

When a tuple appears as the left-hand side of an assignment, type
checking proceeds as if each individual member of the tuple were an
assignment statement to the corresponding member of the complex type on
the right-hand side. If a tuple appears on the right hand side of an
assignment where the left-hand side yields a complex type then the types
of each individual member of the tuple must match the corresponding
types of the complex type exactly. If a tuple is cast into a complex
type then each member of the tuple will be converted into the type of
the corresponding member of the complex type under the rules of
assignment:

\begin{lstlisting}
aggr X
{
      int a;
      byte b;
};

void main()
{
      X x; 
      byte c;
      x = (10, c);      /* Members match exactly */
      x = (X)(10, 1.5); /* float is converted to byte */
}
\end{lstlisting}

\hypertarget{abstract-data-types}{%
\subsection{Abstract data types}\label{abstract-data-types}}

An abstract data type (ADT) defines both storage for members, as in an
aggregate, and the operations that can be performed on that type. Access
to the members of an abstract data type is restricted to enable
information hiding. The scope of the members of an abstract data type
depends on their type. By default access to members that define data is
limited to the member functions. Members can be explicitly exported from
the type using the extern storage class in the member declaration.
Member functions are visible by default, the opposite behavior of data
members. Access to a member function may be restricted to other member
functions by qualifying the declaration with the intern storage class.
The four combinations are:

\begin{lstlisting}
adt Point
{
    int x;                    /* Access only by member fns */
    extern int y;             /* by everybody */
    Point set(Point*);        /* by everybody */
    intern Point tst(Point);  /* only by member fns */
};
\end{lstlisting}

\hypertarget{member-functions-methods}{%
\subsubsection{Member functions
(methods)}\label{member-functions-methods}}

Member functions are defined by type and name. The pair forms a unique
name for the function, so the same member function name can be used in
many types. Using the last example, the member function set could be
defined as:

\begin{lstlisting}
Point Point.set(Point *a)
{
    a->x = 0; /* Set Point value to zero */
    a->y = 0;
    return *a;
}
\end{lstlisting}

An implicit argument of either a pointer to the ADT or the value of the
ADT may be passed to a member function. If the first argument of the
member function declaration in the ADT specification is
\passthrough{\lstinline!* Identifier!} (with the * preceding the name of
the ADT), then a pointer to the ADT is automatically passed as the first
parameter, similarly to the \passthrough{\lstinline!self!} construct in
Smalltalk. If the declaration is of the form
\passthrough{\lstinline!. Identifier!} then the value of the ADT will be
passed to the member function, rather than a pointer to it:

\begin{lstlisting}
adt Point
{
    int x;
    extern int y;
    Point set(*Point); /* Pass &Point as 1st arg */
    Point clr(.Point); /* Pass Point as 1st arg */
    intern Point tst(Point);
};

void func()
{
    Point p;
    p.set(); /* Set receives &p as 1st arg */
}
\end{lstlisting}

\hypertarget{polymorphic-and-parametrized-adts}{%
\subsubsection{Polymorphic and parametrized
ADTs}\label{polymorphic-and-parametrized-adts}}

Alef allows the construction of type-parameterized abstract data types,
similar to generic abstract data types in Ada and Eiffel. An ADT is
parameterized by supplying type parameter names in the declaration. The
type parameters may be used to specify the types of members of the ADT.
The argument type names have the same effect as a typedef to the
polymorphic type. The scope of the types supplied as arguments is the
same as the ADT typename and can therefore be used as a type specifier
in simple declarations. For example the definition of a stack type of
parameter type \passthrough{\lstinline!T!} may be defined as:

\begin{lstlisting}
adt Stack[T]
{
    int tos;
    T data[100];
    void push(*Stack, T);
    T pop(*Stack);
};
\end{lstlisting}

Member functions of Stack are written in terms of the parameter type T.
The implementation of push might be:

\begin{lstlisting}
void Stack.push(Stack *s, T v)
{
    s->data[s->tos++] = v;
}
\end{lstlisting}

\hypertarget{bound-and-unbound-parametrized-adts}{%
\paragraph{Bound and unbound parametrized
ADTs}\label{bound-and-unbound-parametrized-adts}}

The \passthrough{\lstinline!Stack!} type can be instantiated in two
forms. In the bound form, a type is specified for T. The program is type
checked as if the supplied type were substituted for T in the ADT
declaration. For example:

\begin{lstlisting}
Stack[int] stack;
\end{lstlisting}

declares a stack where each element is an int. In the bound form a type
must be supplied for each parameter type.

In the unbound form no parameter types are specified. This allows values
of any type to be stored in the stack. For example:

\begin{lstlisting}
Stack poly;
\end{lstlisting}

declares a stack where each element has polymorphic type.

\hypertarget{conversions-and-promotions}{%
\section{Conversions and Promotions}\label{conversions-and-promotions}}

Alef performs the same implicit conversions and promotions as ANSI C
with the addition of complex type promotion: under assignment, function
parameter evaluation, or function returns, Alef will promote an unnamed
member of a complex type into the type of the left-hand side, formal
parameter, or function.


% !TEX root = ../main.tex 
\hypertarget{lexical-analysis}{%
\chapter{Lexical analysis}\label{lexical-analysis}}

Compilation starts with a preprocessing phase. An ANSI C preprocessor is
used, and any directive valid in such standard is valid in Alef as well.
The preprocessor performs file inclusion and macro substitution.
Comments and lines beginning with the character
\passthrough{\lstinline!\#!}are consumed by the preprocessor. The
preprocessor produces a sequence of tokens for the compiler\footnote{Note
  that using, for example, GCC's CPP on an Alef source file will produce
  a compilation unit with ``\#''-directives, which aren't described in
  this document, but is instead the objective of the compiler engineer
  to manage these implementation-dependent notions.}. Comments are
removed by the preprocessor. Any form of comment accepted by the
preprocessor is valid. For example, GNU's CPP might accept (and remove)
both \passthrough{\lstinline!/*...*/!} and
\passthrough{\lstinline!//..!} style comments.

\hypertarget{tokens}{%
\section{Tokens}\label{tokens}}

The lexical analyzer classifies tokens as identifiers, keywords,
literals and operators. Tokens are separated by white space, which is
ignored in the source except as needed to separate sequences of tokens
which would otherwise be ambiguous. The lexical analyzer is greedy: if
tokens have been consumed up to a given character, then the next token
will be the longest subsequent string of characters that forms a legal
token.

\hypertarget{reserved-words}{%
\subsection{Reserved words}\label{reserved-words}}

\hypertarget{keywords}{%
\subsubsection{Keywords}\label{keywords}}

The following symbols are keywords reserved by the language and may not
be used as user-defined identifiers:

\begin{lstlisting}
adt     aggr    alloc   alt     become    break   check   continue  default
do      else    enum    extern  float     for     goto    if        int
intern  lint    nil     par     proc      raise   rescue  return    sint
sizeof  switch  task    tuple   typedef   typeof  uint    ulint     unalloc
union   usint   void    while   zerox
\end{lstlisting}

Keywords that identify types, such as \passthrough{\lstinline!int!} or
\passthrough{\lstinline!chan!} are, from the point of view of lexical
analysis, seen as identifiers. It will be the job of the compiler to
make it so that at any time, if the parser has to distinguish between an actual
identifier and a typename, the symbol table at that point will
reflect the fact that \passthrough{\lstinline!int!} and
\passthrough{\lstinline!chan!}, for example, are predefined types. In the same
fashion, the \passthrough{\lstinline!nil!} keyword isn't interpreted as
a keyword, but as an always-defined constant of type
\passthrough{\lstinline!void\\*!} of value
\passthrough{\lstinline!0!}. In other words, to correctly build the parse tree
the parser needs the information to discern identifiers and typenames (otherwise, 
type declarations such as \passthrough{\lstinline!int* a!} would be undiscernible 
from multiplication expressions, as both would be seen as 
\passthrough{\lstinline!identifier op_mul identifier!}.)

\hypertarget{delimiters-and-operators}{%
\subsubsection{Delimiters and operators}\label{delimiters-and-operators}}

The following symbols are used as delimiters and operators in the
language:

\begin{lstlisting}
+   -   /   =   >   <   !   \%   &   |   ?   .   "   '   {   } 
[   ]   (   )   \*   ;   :   ^   +=  -=  /=  \*=  \%=  &=  |=  ^=
<<= >>= ==  !=  --  <-  ->  ++  ::  :=
\end{lstlisting}

\hypertarget{identifiers}{%
\subsection{Identifiers}\label{identifiers}}

An identifier, also called a lexical name, is any sequence of
alpha-numeric characters and the underscore character
\passthrough{\lstinline!\\\_!}. An identifiers binds a name to a
semantic object such as a type, a function or a variable. Identifiers
starting with ``ALEF'' are reseved for use in the runtime system.

\begin{lstlisting}
ASCII      = "a" | "b" | ... | "z" | "A" | "B" | ... | "Z" .
Digits     = "0" | "1" | ... | "9" .
Identifier = ( "_" | ASCII ) { ( "_" | ASCII | Digits ) } .
\end{lstlisting}

Identifiers may define variables, types, functions, function prototypes
or enumerators. An identifier has associated a
\protect\hyperlink{scopes}{scope} and a \protect\hyperlink{storage-classes}{storage classes}.

\hypertarget{literals}{%
\subsection{Literals}\label{literals}}

Alef literals are integer and floating point numbers, characters,
strings and runestrings (UTF-8 strings.) There are five types of
constant:

\begin{lstlisting}
Literal       = StringLit | RunestringLit | CharLit | IntLit | FloatLit .
StringLit     = '"' { ASCII | Escapes } '"' .
Escapes       = "\" ("0" | "n" | "r" | "t" | "b" | "f" | "a" | "v" | "\" | `"` ) .
RunestringLit = @TODO
CharLit       = @TODO .
IntLit        = [ "0" [ ( HexLit | OctalLit ) ] ] | DecimalLit .
HexLit        = ("x" | "X") { HexDigits }.
HexDigits     = "0" ... "9" | "A" ... "F" | "a" ... "f" .
OctalLit      = { OctalDigits } .
OctalDigits   = "0" ... "7"
DecimalLit    = ( "1" | ... | "9" ) { Digits } .
FloatLit      = @TODO . 
\end{lstlisting}

Character literals have the type \passthrough{\lstinline!uint!} and can
hold UTF-8 code points. String literals have type
\passthrough{\lstinline!static array of byte!} and are NUL
(\passthrough{\lstinline!\\0!}) terminated (appended by the compiler);
therefore, the sizeof operator applied to a string yields the number of
bytes including the appended NUL. Rune string literals are sequences of
UTF-8 code points and have type
\passthrough{\lstinline!static array of uint!}, and are NUL
(\passthrough{\lstinline!U+0000!}) terminated (appended by the
compiler); therefore, the sizeof operator applied to a string yields the
number of runes in the runestring, in terms of
\passthrough{\lstinline!sizeof(uint)!} including the appended NUL. The
following table shows valid characters after an escape and the value of
the constant:

\begin{lstlisting}
0     NUL     Null character
n     NL      Newline
r     CR      Carriage return
t     HT      Horizontal tab
b     BS      Backspace
f     FF      Form feed
a     BEL     Beep
v     VT      Vertical tab
\         \   Backslash
"     "   Double quote
\end{lstlisting}

Float literals have type \passthrough{\lstinline!float!}. Integer
literal have type \passthrough{\lstinline!int!}.


% !TEX root = ../main.tex 

\hypertarget{declarations}{%
\chapter{Declarations}\label{declarations}}

\hypertarget{programs}{%
\section{Programs}\label{programs}}

A declaration attaches a type to an identifier; it need not reserve
storage. A declaration which reserves storage is called a definition. A
program consists of a list of declarations. A declaration can define a
simple variable, a function, a prototype to a function, an ADT function,
a type specification, or a type definition.

\begin{lstlisting}
Program     = { Declaration } .
Declaration = [ Visibility ] ( SimpleDecl | ComplexDecl | TypeDefs ).
Visibility  = "intern" | "extern" .
\end{lstlisting}

A declaration introduces an identifier and specifies its type. A
definition is a declaration that also reserves storage for an
identifier. An object is an area of memory of known type produced by a
definition. Function prototypes, variable declarations preceded by
extern, and type specifiers are declarations. Function declarations with
bodies and variable declarations are examples of definitions.

\hypertarget{scopes}{%
\subsection{Scopes}\label{scopes}}

Scopes define where a named object, that is an identifier with a precise
semantic meaning, can be referenced.

\hypertarget{file-scope}{%
\subsubsection{File scope}\label{file-scope}}

A declaration introduces an identifier and specifies its type. A
definition is a declaration that also reserves storage for an
identifier. Declaration at the file scope are implicitly assumed to be
``extern'' declarations, that is, visible to all compilations units. If
a declaration is preceded with the ``intern'' keyword, its scope is
narrowed to its compilation unit only. Variable declarations (that is,
variable declarations preceded by the keyword ``extern'') aren't
definitions, and therefore do not allocate space for the variable.
Similarly, function prototypes do not allocate space\footnote{\textbf{@TODO}:
  The use of extern as a keyword to make a variable visible in the
  entire compilation unit and to make non-allocating definitions is
  highly unclear -- we might use a different keyword to indicate
  non-allocating definitions.}.

\hypertarget{type-scope}{%
\subsubsection{Type scope}\label{type-scope}}

Members of complex types (ADT's, aggregates or union) are in scope only
when an access operator is applied to objects of their appartaining
type. Members of unions and aggregates are always accessible from
outside. Access to members of ADT's can be restricted (or enabled) using
visibility specifiers ``intern'' and ``extern'': variable members are,
by default, ``intern'', that is not accessible from outside. Methods
(function members of an ADT) are by default ``extern''.

\hypertarget{function-scope}{%
\subsubsection{Function scope}\label{function-scope}}

Labels and raise statement's identifiers can be referenced from the
start of a function to its end, regardless of the position of the
declaration.

\hypertarget{local-scope}{%
\subsubsection{Local scope}\label{local-scope}}

Local identifiers are declared at the start of a block\footnote{\textbf{@TODO}:
  This means that, in a function's body, automatic variables can be
  declared only at the start, before any statement.}. A local identifier
has scope starting from its declaration to the end of the block in which
it was declared.

\hypertarget{storage-classes}{%
\subsection{Storage classes}\label{storage-classes}}

While scopes define where identifiers have meaning, storage classes
define the lifetime ``of the meaning'', that is, when variables and
functions are created and deleted, and to what value they're
initialized.

\hypertarget{automatic-storage-class}{%
\subsubsection{Automatic storage class}\label{automatic-storage-class}}

Automatic objects are created at the entry of the block in which they
were declared, and their value is undefined upon creation.

\hypertarget{parameter-storage-class}{%
\subsubsection{Parameter storage class}\label{parameter-storage-class}}

Function parameters are created by function invocation and are destroyed
at function exit. They have the value of the values passed by the
caller.

\hypertarget{static-storage-class}{%
\subsubsection{Static storage class}\label{static-storage-class}}

Static objects exist from invocation of the program until termination,
and uninitialized static objects have, at creation, the value 0.

\hypertarget{simple-declarations}{%
\section{Simple declarations}\label{simple-declarations}}

A simple declaration consists of a type specifier and a list of
identifiers. Each identifier may be qualified by deriving operators.
Simple declarations at the file scope may be initialized. Function
pointer declarations have \emph{per sé} rules.

\begin{lstlisting}
SimpleDecl = Type [ PtrSpec ] ( FuncPtr | BaseDecl ) . 
FuncPtr    = "(" [ PtrSpec ] Identifier ( FuncPtrFuncDecl | FuncPtrVarDecl ) .
BaseDecl   = Identifier ( FuncDecl | VarDecl |  MethDecl ) .
\end{lstlisting}

\hypertarget{function-pointer-declarations}{%
\subsection{Function pointer
declarations}\label{function-pointer-declarations}}

\begin{lstlisting}
FuncPtrFuncDecl  = "(" [ ParamList ] ")" ")" "(" [ ParamList ] ")" ( ";" | Block ) .
ParamList         = Param { "," Param } .
Param            = SimpleParam | TupleParam  | "..." .
SimpleParam      = BaseType [ PtrSpec ] [ ( BaseParam |  FuncPtrParam ) ] . 
BaseParam        = Identifier [ ArraySpec ] . 
FuncPtrParam     = "(" PtrSpec [ Identifier ] [ ArraySpec ] ")"  "(" [ ParamList ] ")" .
TupleParam       = "tuple" "(" TupleList ")" [ [ PtrSpec ] ( BaseParam | FuncPtrParam ) ] .
FuncPtrVarDecl   = [ ArraySpec ] ")"  "(" [ ParamList ] ")" [ "=" InitExpression ] ( ";" |  "," [ PtrSpec ] ( "(" [ PtrSpec ] Identifier FuncPtrVarDecl | Identifier VarDecl ) ) .
\end{lstlisting}

The parameters received by a function taking variable arguments are
referenced using the ellipsis \passthrough{\lstinline!...!}. The token
\passthrough{\lstinline!...!} yields is a value of type
\passthrough{\lstinline!pointer to void!}. The value points at the first
location after the formal parameters.

\hypertarget{examples-of-function-pointer-declarations.}{%
\subsubsection{Examples of function pointer
declarations.}\label{examples-of-function-pointer-declarations.}}

\hypertarget{a-variable-of-type-function-pointer.}{%
\paragraph{A variable of type array of function pointer.}
\label{a-variable-of-type-function-pointer.}}
In this example,
\begin{lstlisting}
int * (* func_ptr[2] ) (int, bool);
\end{lstlisting}

\hypertarget{a-function-returning-a-function.}{%
\paragraph{A function returning a
function.}\label{a-function-returning-a-function.}}
In this example,

\begin{lstlisting}
int * (func_func (float, char)) (int, bool);
\end{lstlisting}

\hypertarget{a-notable-case}{%
\paragraph{A notable case}\label{a-notable-case}}
In this example,

\begin{lstlisting}
int * (func) (int, bool);
\end{lstlisting}

Is interpreted as a function, not as a variable of type pointer to a
function.

\hypertarget{variable-function-and-method-declarations}{%
\subsection{Variable, function and method
declarations}\label{variable-function-and-method-declarations}}

\begin{lstlisting}
FuncDecl = "(" [ ParamList ] ")" ( ";" | Block ) .
VarDecl  = [ ArraySpec ] [ "=" InitExpression ] ( ";" | "," [ PtrSpec ] ( "(" [ PtrSpec ] Identifier FuncPtrVarDecl | Identifier VarDecl ) .
MethDecl = "." Identifier "(" [ ParamList ] ")" Block .
\end{lstlisting}

\hypertarget{initializers}{%
\subsubsection{Initializers}\label{initializers}}
Only simple declarations at the file scope may be initialized\footnote{We may want to change this.}. 
An initialization consists of a constant expression or a list of
constant expressions separated by commas and enclosed by braces. An
array or complex type requires an explicit set of braces for each level
of nesting. Unions may not be initialized. All the components of a
variable need not be explicitly initialized; uninitialized elements are
set to zero. ADT types are initialized in the same way as aggregates
with the exception of ADT function members which are ignored for the
purposes of initialization. Elements of sparse arrays can be initialized
by supplying a bracketed index for an element. Successive elements
without the index notation continue to initialize the array in sequence.
For example:

\begin{lstlisting}
byte a[256] = {
    ['a'] 'A',    /* Set element 97 to 65 */
    ['a'+1] 'B',  /* Set element 98 to 66 */
    'C'           /* Set element 99 to 67 */
};
\end{lstlisting}

If the dimensions of the array are omitted from the array-spec the
compiler sets the size of each dimension to be large enough to
accommodate the initialization. The size of the array in bytes can be
found using sizeof.

\begin{lstlisting}
InitExpression     = Expression | ArrayElementInit | MemberInit | BlockInit .
ArrayElementInit   = "[" Expression "]"  ( Expression | BlockInit ) .
MemberInit         = "." Identifier Expression .
BlockInit = "{" [ InitExpressionList ] "}" .
InitExpressionList = InitExpression [ "," InitExpression ] .
\end{lstlisting}

\hypertarget{complex-type-declaration}{%
\section{Complex type declaration}\label{complex-type-declaration}}

Complex declarations define new aggregates, unions, ADT's and enums in
the innermost active scope.

\begin{lstlisting}
ComplexDecl = ( AggrDecl | UnionDecl | AdtDecl | EnumDecl ) ";" .
\end{lstlisting}

\hypertarget{unions-and-aggregates}{%
\subsection{Unions and aggregates}\label{unions-and-aggregates}}

\begin{lstlisting}
AggrDecl        = "aggr" [ Identifier ] "{" { AggrUnionMember } "}" [ Identifier ] .
UnionDecl       = "union" [ Identifier ] "{" { AggrUnionMember } "}" [ Identifier ] .
AggrUnionMember = ComplexDefs | VarMember .
VarMember       = Type [ [ PtrSpec ]  ( SimpleMember | FuncPtrMember ) { "," [ PtrSpec ] ( SimpleMember | FuncPtrMember ) } ] ";" .
SimpleMember    = Identifier [ ArraySpec ] .
FuncPtrMember   = "(" [ PtrSpec ] Identifier [ ArraySpec ] ")" "(" [ParamList] ")" .
\end{lstlisting}

\hypertarget{abstract-data-types-1}{%
\subsection{Abstract data types}\label{abstract-data-types-1}}

\begin{lstlisting}
AdtDecl                = "adt" [ Identifier ] [ AdtGenSpec ] "{" { AdtMember } "}" [ Identifier ] .
AdtGenSpec             = "[" Identifier { "," Identifier "}" "]" .
AdtMember              = [ Visibility ] Type [ [ PtrSPec ] ( AdtFuncPtrMember | AdtBaseMember ) ] ";" .
AdtFuncPtrMember       = "(" [ PtrSpec ] Identifier ( AdtFuncPtrMethodMember | AdtFuncPtrVarMember ) .
AdtFuncPtrMethodMember = "(" [ AdtMethodRefParam [ "," ParamList ] ] | ParamList ")" ")" "(" [ ParamList ] ")" .
AdtFuncPtrVarMember    = [ ArraySpec ] ")" "(" [ ParamList ] ")"  [ "," [ PtrSpec ] ( "(" [ PtrSpec ] Identifier AdtFuncPtrVarMember | Identifier AdtVarMember ) ] .
AdtBaseMember          = Identifier ( AdtMethodMember | AdtVarMember ) .
AdtMethodMember        = "(" [ AdtMethodRefParam [ "," ParamList ] ] | ParamList ")" .
AdtMethodRefParam      = ( "*" | "." ) Identifier [ Identifier ] .
AdtVarMember           = [ ArraySpec ] [ "," [ PtrSpec ] ( "(" [ PtrSpec ] Identifier AdtFuncPtrVarMember | Identifier AdtVarMember ) ] .
\end{lstlisting}

\hypertarget{enumerators}{%
\subsection{Enumerators}\label{enumerators}}

\begin{lstlisting}
EnumDecl   = "enum" [ Identifier ] "{" { EnumMember } "}" .
EnumMember = Identifier [ "=" Expression ]
\end{lstlisting}

\hypertarget{type-definitions}{%
\section{Type definitions}\label{type-definitions}}

Type definitions are declarations which start with the keyword
``typedef''. Type definitions can introduce new polymorphic variables in
the innermost active scope, forward references to complex types and new
names for basic and derived types.

\begin{lstlisting}
TypeDefs       = "typedef" ( PolyVarTypeDef | ForwardDef )
PolyVarTypeDef = BaseType [ PtrSpec ] ( DerivedTypeDef | FuncPtrTypeDef ) ";" .
DerivedTypeDef = Identifier [ ArraySpec ] .
FuncPtrTypeDef = "(" [ PtrSpec ] Identifier [ ArraySpec ] ")" "(" [ParamList] ")" .
ForwardDef     = ( "aggr" | "union" | "adt" ) Identifier ";" .
\end{lstlisting}

To declare complex types with mutually dependent pointers, it is
necessary to use a typedef to predefine one of the types. Alef does not
permit mutually dependent complex types, only references between them.
For example:

\begin{lstlisting}
typedef aggr A;
aggr B
{
    A *aptr;
    B *bptr;
};
aggr A
{
    A *aptr;
    B *bptr;
};
\end{lstlisting}

%\hypertarget{examples-of-type-definitions}{%
%\subsection{Examples of type
%definitions:}\label{examples-of-type-definitions}}
%
%\hypertarget{a-polymorphic-definition.}{%
%\subsubsection{A polymorphic
%definition.}\label{a-polymorphic-definition.}}
%
%\begin{lstlisting}
%typedef A;
%(TypeDefs => "typedef" PolyVarTypeDef => "typedef" Identifier ";" => "typedef" "A" ";")
%\end{lstlisting}
%
%\hypertarget{a-derived-type-definition.}{%
%\subsubsection{A derived type
%definition.}\label{a-derived-type-definition.}}
%
%\begin{lstlisting}
%typedef int * array_ptr_to_int [];  
%(TypeDefs => "typedef" PolyVarTypeDef => "typedef" Identifier DerivedTypeDef => 
%  => "typedef" Identifier PtrSpec Identifier ArraySpec => 
%    =*> "typedef" "int" "*" "array_ptr_to_int" "[]" ";")
%\end{lstlisting}
%
%\hypertarget{a-function-pointer-type-definition.}{%
%\subsubsection{A function pointer type
%definition.}\label{a-function-pointer-type-definition.}}
%
%\begin{lstlisting}
%typedef int (* array_ptr_to_func () ) (int, float, char) ";"
%\end{lstlisting}
%

% !Tex root = ../main.tex
\hypertarget{expressions}{%
\chapter{Expressions}\label{expressions}}

The order of expression evaluation is not defined except where noted.
That is, unless the definition of the operator guarantees evaluation
order, an operator may evaluate any of its operands first. The behavior
of exceptional conditions such as divide by zero, arithmetic overflow,
and floating point exceptions is not defined by the specification and is
implementation dependent.

\hypertarget{pointer-generation}{%
\section{Pointer generation}\label{pointer-generation}}

References to expressions of type
\passthrough{\lstinline!function returning T!} and
\passthrough{\lstinline!array of T!} are rewritten to produce pointers
to either the function or the first element of the array. That is,
\passthrough{\lstinline!function returning T!} becomes
\passthrough{\lstinline!pointer to function returning T!} and
\passthrough{\lstinline!array of T!} becomes
\passthrough{\lstinline!pointer to the first element of array of T!}.

\hypertarget{primary-expressions}{%
\section{Primary expressions}\label{primary-expressions}}

Primary expressions are identifiers, constants, or parenthesized
expressions:

\begin{lstlisting}
PrimaryExpression = Identifier | Literal | "nil" | [ "tuple" ] "(" ExpressionList ")" .
\end{lstlisting}

The primary expression \passthrough{\lstinline!nil!} returns a pointer
of type \passthrough{\lstinline!pointer to void!} of value 0 which is
guaranteed not to point at an object. \passthrough{\lstinline!nil!} may
also be used to initialize channels and polymorphic types to a known
value. The only legal operation on these types after such an assignment
is a test with one of the equality test operators and the
\passthrough{\lstinline!nil!} value.

\hypertarget{postfix-expressions}{%
\section{Postfix expressions}\label{postfix-expressions}}

\begin{lstlisting}
PostfixExpression = ( PrimaryExpression | AdtNameCall ) { PostfixOperand } .
AdtNameCall       = "." Typename "." Identifier .
PostfixOperand    = ArrayAccess | FuncCall | MemberAccess | IndirectAccess | UnaryPostfix .
ArrayAccess       = "[" Expression "]" .
FuncCall          = "(" [ ExpressionList ] ")" .
MemberAccess      = "." Identifier .
IndirectAccess    = "->" Identifier .
UnaryPostfix      = "++" | "--" | "?" .
ExpressionList    = Expression { "," Expression } .
\end{lstlisting}

\hypertarget{array-reference}{%
\subsection{Array reference}\label{array-reference}}

A primary expression followed by an expression enclosed in square
brackets is an array indexing operation. The expression is rewritten to
be

\begin{lstlisting}
*((PrimaryExpression)+(Expression))
\end{lstlisting}

One of the expressions must be of type pointer, the other of integral
type.

\hypertarget{function-calls}{%
\subsection{Function calls}\label{function-calls}}

Function call postfix operators yield a value of type
\passthrough{\lstinline!pointer to function!}. A type declaration for
the function must be declared prior to a function call. The declaration
can be either the definition of the function or a function prototype.
The types of each argument in the prototype must match the corresponding
expression type under the rules of promotion and conversion for
assignment.

\hypertarget{function-call-promotions}{%
\subsubsection{Function call
promotions}\label{function-call-promotions}}

In addition, unnamed complex type members will be promoted
automatically. For example:

\begin{lstlisting}
aggr Test
{
    int t;
    Lock; /* Unnamed substructure */
};

Test yuk;   /* Definition of complex object yuk */

void lock(Lock*); /* Prototype for function lock */

void main()
{
    lock(&yuk); /* address of yuk.Lock is passed */
}
\end{lstlisting}

\hypertarget{adt-namecalls}{%
\subsubsection{ADT namecalls}\label{adt-namecalls}}

Calls to member functions may use the type name instead of an expression
to identify the ADT. If the function has an implicit first parameter,
\passthrough{\lstinline!nil!} is passed. Given the following definition
of \passthrough{\lstinline!X!} these two calls are equivalent:

\begin{lstlisting}
adt X
{
    int i;
    void f(*X);
};

X val;

((X*)nil)->f();

.X.f();
\end{lstlisting}

This form is illegal if the implicit parameter is declared by value
rather than by reference.

\hypertarget{polymorphic-promotions}{%
\subsubsection{Polymorphic promotions}\label{polymorphic-promotions}}

Calls to member functions of polymorphic ADT's have special promotion
rules for function arguments. If a polymorphic type
\passthrough{\lstinline!P!} has been bound to an actual type
\passthrough{\lstinline!T!} then an actual parameter
\passthrough{\lstinline!v!} of type \passthrough{\lstinline!T!}
corresponding to a formal parameter of type \passthrough{\lstinline!P!}
will be promoted into type \passthrough{\lstinline!P!} automatically.
The promotion is equivalent to \passthrough{\lstinline!(alloc P)v!} as
described in the Casts section. For example:

\begin{lstlisting}
adt X[T]
{
    void f(*X, T);
};

X[int] bound;

bound.f(3);           /* 3 is promoted as if (alloc T)3 */
bound.f((alloc T)3);  /* illegal: int not same as poly */
\end{lstlisting}

In the unbound case values must be explicitly converted into the
polymorphic type using the cast syntax:

\begin{lstlisting}
X unbound;

unbound.f((alloc T)3);  /* 3 is converted into poly */
unbound.f(3);           /* illegal: int not same as poly */
\end{lstlisting}

In either case, the actual parameter must have the same type as the
formal parameter after any binding has taken place.

\hypertarget{complex-type-references}{%
\subsection{Complex type references}\label{complex-type-references}}

The operator \passthrough{\lstinline!.!} references a member of a
complex type. The first part of the expression must yield
\passthrough{\lstinline!union!}, \passthrough{\lstinline!aggr!}, or
\passthrough{\lstinline!adt!}. Named members must be specified by name,
unnamed members by type. Only one unnamed member of type typename is
permitted in the complex type when referencing members by type,
otherwise the reference would be ambiguous.

If the reference is by typename and no members of typename exist in the
complex, unnamed substructures will be searched breadth first. The
operation \passthrough{\lstinline!$->$!} uses a pointer to reference a
complex type member. The \passthrough{\lstinline!$->$!} operator follows
the same search and type rules as \passthrough{\lstinline!.!} and is
equivalent to the expression
\passthrough{\lstinline!(*PostfixExpression).tag.!}

References to polymorphic members of unbound polymorphic ADT's behave as
normal members: they yield an unbound polymorphic type. Bound
polymorphic ADT's have special rules. Consider a polymorphic type
\passthrough{\lstinline!P!} that is bound to an actual type
\passthrough{\lstinline!T!}. If a reference to a member or function
return value of type \passthrough{\lstinline!P!} is assigned to a
variable \passthrough{\lstinline!v!} of type \passthrough{\lstinline!T!}
using the assignment operator \passthrough{\lstinline!=!}, then the type
of \passthrough{\lstinline!P!} will be narrowed to
\passthrough{\lstinline!T!}, assigned to \passthrough{\lstinline!v!},
and the storage used by the polymorphic value will be unallocated. The
value assignment operator \passthrough{\lstinline!:=!} performs the same
type narrowing but does not unallocate the storage used by the
polymorphic value. For example:

\begin{lstlisting}
adt Stack[T]
{
    int tos;
    T data[100];
};

Stack[int] s;
int i, j, k;

i := s.data[s->tos];
j = s.data[s->tos];
k = s.data[s->tos]; /* s.data[s->tos] has been unallocated. */
\end{lstlisting}

The first assignment copies the value at the top of the stack into
\passthrough{\lstinline!i!} without altering the data structure. The
second assignment moves the value into \passthrough{\lstinline!j!} and
unallocates the storage used in the stack data structure. The third
assignment is illegal since \passthrough{\lstinline!data[s->tos]!} has
been unallocated.

\hypertarget{postfix-increment-and-decrement}{%
\subsection{Postfix increment and
decrement}\label{postfix-increment-and-decrement}}

The postfix increment ( \passthrough{\lstinline!++!} ) and decrement (
\passthrough{\lstinline!--!} ) operators return the value of expression,
then increment it or decrement it by 1. The expression must be an
l-value of integral or pointer type.

\hypertarget{prefix-expressions}{%
\section{Prefix expressions}\label{prefix-expressions}}

\begin{lstlisting}
UnaryExpression = PostfixExpression | UnaryPrefix | CastPrefix .
UnaryPrefix     = ( "<-" | "++" | "--" | "zerox" ) UnaryExpression .
CastPrefix      = UnaryOperator Term .
UnaryOperator   = ( "?" | "*" | "&" | "!" | "+" | "-" | "~" | "sizeof" ) .
\end{lstlisting}

\hypertarget{prefix-increment-and-decrement}{%
\subsection{Prefix increment and
decrement}\label{prefix-increment-and-decrement}}

The prefix increment ( \passthrough{\lstinline!++!} ) and prefix
decrement ( \passthrough{\lstinline!--!} ) operators add or subtract one
to a unary expression and return the new value. The unary expression
must be an l-value of integral or pointer type.

\hypertarget{receive-and-can-receive}{%
\subsection{Receive and can receive}\label{receive-and-can-receive}}

The prefix operator \passthrough{\lstinline!<-!} receives a value from a
channel. The unary expression must be of type
\passthrough{\lstinline!channel of T!}. The type of the result will be
\passthrough{\lstinline!T!}. A process or task will block until a value
is available from the channel. The \emph{prefix} operator
\passthrough{\lstinline!?!} returns \passthrough{\lstinline!1!} if a
channel has a value available for receive, \passthrough{\lstinline!0!}
otherwise.

\hypertarget{send-and-can-send}{%
\subsection{Send and Can send}\label{send-and-can-send}}

The postfix operator \passthrough{\lstinline!<-!}, on the left-hand side
of an assignment sends a value to a channel, for example:

\begin{lstlisting}
chan(int) c;
c <-= 1;    /* send 1 on channel c */
\end{lstlisting}

The \emph{postfix} operator \passthrough{\lstinline!?!} returns
\passthrough{\lstinline!1!} if a thread can send on a channel without
blocking, \passthrough{\lstinline!0!} otherwise. The prefix or postfix
blocking test operator ? is only reliable when used on a channel shared
between tasks in a single process. A process may block after a
successful \passthrough{\lstinline!?!} because there may be a race
between processes competing for the same channel.

\hypertarget{indirection}{%
\subsection{Indirection}\label{indirection}}

The unary prefix operator \passthrough{\lstinline!*!} retrieves the
value pointed to by its operand. The operand must be of type
\passthrough{\lstinline!pointer to T!}. The result of the indirection is
a value of type \passthrough{\lstinline!T!}.

\hypertarget{unary-plus-and-minus}{%
\subsection{Unary plus and minus}\label{unary-plus-and-minus}}

Unary plus is equivalent to
\passthrough{\lstinline!(0+(UnaryExpression))!}. Unary minus is
equivalent to \passthrough{\lstinline!(0-(UnaryExpression))!}. An
integral operand undergoes integral promotion. The result has the type
of the promoted operand.

\hypertarget{bitwise-negate}{%
\subsection{Bitwise negate}\label{bitwise-negate}}

The operator \passthrough{\lstinline!\~!} performs a bitwise negation of
its operand, which must be of integral type.

\hypertarget{logical-negate}{%
\subsection{Logical negate}\label{logical-negate}}

The operator \passthrough{\lstinline"!"} performs logical negation of
its operand, which must of arithmetic or pointer type. If the operand is
a pointer and its value is \passthrough{\lstinline!nil!} the result is
integer \passthrough{\lstinline!1!}, otherwise
\passthrough{\lstinline!0!}. If the operand is arithmetic and the value
is \passthrough{\lstinline!0!} the result is
\passthrough{\lstinline!1!}, otherwise the result is
\passthrough{\lstinline!0!}.

\hypertarget{zerox}{%
\subsection{Zerox}\label{zerox}}

The \passthrough{\lstinline!zerox!} operator may only be applied to an
expression of polymorphic type. The result of
\passthrough{\lstinline!zerox!} is a new fat pointer, which points at a
copy of the result of evaluating its operand. For example:

\begin{lstlisting}
typedef Poly;
Poly a, b, c;
a = (alloc Poly)10;
b = a; 
c = zerox a;
\end{lstlisting}

causes \passthrough{\lstinline!a!} and \passthrough{\lstinline!b!} to
point to the same storage for the value \passthrough{\lstinline!10!} and
\passthrough{\lstinline!c!} to point to distinct storage containing
another copy of the value \passthrough{\lstinline!10!}.

\hypertarget{sizeof-operator}{%
\subsection{Sizeof operator}\label{sizeof-operator}}

The \passthrough{\lstinline!sizeof!} operator yields the size in
\passthrough{\lstinline!byte!}s of its operand, which may be an
expression or the parenthesized name of a type. The size is determined
from the type of the operand, which is not itself evaluated. The result
is a \passthrough{\lstinline!signed integer!} constant.

\hypertarget{term-expressions}{%
\section{Term expressions}\label{term-expressions}}

\begin{lstlisting}
Term            = UnaryExpression | CastExpression | AllocExpression .
CastExpression  = "(" TypeCast ")" Term .
AllocExpression = "(" "alloc" Typename ")" Term .
TypeCast        = BaseType [ PtrSpec ] [ FuncCast ] | "tuple" "(" TupleList ")" .
FuncCast        = "(" [ PtrSpec ] ")" "(" [ ParamList ] ")" .
\end{lstlisting}

\hypertarget{cast-expressions}{%
\subsection{Cast expressions}\label{cast-expressions}}

A cast converts the result of an expression into a new type. A value of
any type may be converted into a polymorphic type by adding the keyword
\passthrough{\lstinline!alloc!} before the polymorphic type name. This
has the effect of allocating storage for the value, assigning the value
of the operand into the storage, and yielding a fat pointer as the
result. For example, to create a polymorphic variable with integer value
\passthrough{\lstinline!10!}:

\begin{lstlisting}
typedef Poly;
Poly p;
p = (alloc Poly) 10;
\end{lstlisting}

The only other legal cast involving a polymorphic type converts one
polyname into another.

\hypertarget{binary-expressions}{%
\section{Binary expressions}\label{binary-expressions}}

Binary operators in LL grammars lose their left associativity. A given
implementation will use Dijkstra's shunting yard algorithm or a Pratt
Parser. Nonetheless, a list of valid expressions follows.

\begin{lstlisting}
Expression   = Term | Expression BinaryOp Expression .
BinaryOp     = SumOp | MulOp | LogOp | ShOp | CompOp | EqOp | AssOp | IterOp .
SumOp        = "+" | "-" .
MulOp        = "\*" | "/" .
LogOp        = BitwiseLogOp | "&&" | "||" .
BitwiseLogOp = "^" | "&" | "|" .
ShOp         = "<<" | ">>" .
CompOp       = "<" | ">" | ">= " | "<=" .
EqOp         = "==" | "!=" . 
AssOp        = "=" | ":=" | "<-=" | (SumOp | MulOp | BitwiseLogOp | ShOp ) "=" .
IterOp       = "::" .
\end{lstlisting}

\hypertarget{multiply-divide-and-modulus}{%
\subsection{Multiply, divide and
modulus}\label{multiply-divide-and-modulus}}

The operands of \passthrough{\lstinline!*!} and
\passthrough{\lstinline!/!} must have arithmetic type. The operands of
\passthrough{\lstinline!\%!} must be of integral type. The operator
\passthrough{\lstinline!/!} yields the quotient,
\passthrough{\lstinline!\%!} the remainder, and
\passthrough{\lstinline!*!} the product of the operands. If
\passthrough{\lstinline!b!} is non-zero then
\passthrough{\lstinline!a == (a/b) + a\%b!} should always be true.

\hypertarget{add-and-subtract}{%
\subsection{Add and subtract}\label{add-and-subtract}}

The \passthrough{\lstinline!+!} operator computes the sum of its
operands. Either one of the operands may be a pointer. If
\passthrough{\lstinline!P!} is an expression yielding a pointer to type
\passthrough{\lstinline!T!} then \passthrough{\lstinline!P+n!} is the
same as \passthrough{\lstinline!p+(sizeof(T)*n)!}. The
\passthrough{\lstinline!-!} operator computes the difference of its
operands. The first operand may be of pointer or arithmetic type. The
second operand must be of arithmetic type. If
\passthrough{\lstinline!P!} is an expression yielding a pointer of type
\passthrough{\lstinline!T!} then \passthrough{\lstinline!P-n!} is the
same as \passthrough{\lstinline!p-(sizeof(T)*n)!}. Thus if
\passthrough{\lstinline!P!} is a pointer to an element of an array,
\passthrough{\lstinline!P+1!} will point to the next object in the array
and \passthrough{\lstinline!P-1!} will point to the previous object in
the array.

\hypertarget{shift-operators}{%
\subsection{Shift operators}\label{shift-operators}}

The shift operators perform bitwise shifts. If the first operand is
unsigned, \passthrough{\lstinline!<<!} performs a logical left shift by
a number of bits as its right operand. If the first operand is signed,
\passthrough{\lstinline!<<!} performs an arithmetic left shift by a
number of bits as its right operand. The left operand must be of
integral type. The \passthrough{\lstinline!>>!} operator is a right
shift and follows the same rules as left shift.

\hypertarget{relational-operators}{%
\subsection{Relational Operators}\label{relational-operators}}

The values of expressions can be compared using relational operators.
The operators are \passthrough{\lstinline!<!} (less than),
\passthrough{\lstinline!>!} (greater than), \passthrough{\lstinline!<=!}
(less than or equal to) and \passthrough{\lstinline!>=!} (greater than
or equal to). The operands must be of arithmetic or pointer type. The
value of the expression is \passthrough{\lstinline!1!} if the relation
is true, otherwise \passthrough{\lstinline!0!}. The usual arithmetic
conversions are performed. Pointers may only be compared to pointers of
the same type or of type \passthrough{\lstinline!void*!}.

\hypertarget{equality-operators}{%
\subsection{Equality operators}\label{equality-operators}}

The operators \passthrough{\lstinline!==!} (equal to) and
\passthrough{\lstinline"!="} (not equal) follow the same rules as
relational operators. The equality operations may be applied to
expressions yielding channels and polymorphic types for comparison with
the value \passthrough{\lstinline!nil!}. A pointer of value
\passthrough{\lstinline!nil!} or type \passthrough{\lstinline!void*!}
may be compared to any pointer.

\hypertarget{bitwise-logic-operators}{%
\subsection{Bitwise logic operators}\label{bitwise-logic-operators}}

The bitwise logic operators perform bitwise logical operations and apply
only to integral types. The operators are \passthrough{\lstinline!\&!}
(bitwise and), \passthrough{\lstinline!\^!} (bitwise exclusive or) and
\passthrough{\lstinline!|!} (bitwise inclusive or).

\hypertarget{logical-operators}{%
\subsection{Logical operators}\label{logical-operators}}

The \passthrough{\lstinline!\&\&!} operator returns
\passthrough{\lstinline!1!} if both of its operands evaluate to
non-zero, otherwise \passthrough{\lstinline!0!}. The
\passthrough{\lstinline!||!} operator returns
\passthrough{\lstinline!1!} if either of its operand evaluates to
non-zero, otherwise \passthrough{\lstinline!0!}. Both operators are
guaranteed to evaluate strictly left to right. Evaluation of the
expression will cease as soon the final value is determined. The
operands can be any mix of arithmetic and pointer types.

\hypertarget{constant-expressions}{%
\subsection{Constant expressions}\label{constant-expressions}}

A constant expression is an expression which can be fully evaluated by
the compiler during translation rather than at runtime.
Constant expression appears as part of initialization, channel buffer
specifications, and array dimensions. The following operators may not be
part of a constant expression: function calls, assignment, send,
receive, increment and decrement. Address computations using the
\passthrough{\lstinline!\&!} (address of) operator on static
declarations is permitted.

\hypertarget{assignment}{%
\subsection{Assignment}\label{assignment}}

The assignment operators are:

\begin{lstlisting}
= := += *= /= -= %= &= |= ^= >>= <<= 
\end{lstlisting}

The left side of the expression must be an l-value. Compound assignment
allows the members of a complex type to be assigned from a member list
in a single statement. A compound assignment is formed by casting a
tuple into the complex type. Each element of the tuple is evaluated in
turn and assigned to its corresponding element in the complex types. The
usual conversions are performed for each assignment.

\begin{lstlisting}
/* Encoding of read message to send to file system */
aggr Readmsg
{
    int fd;
    void *data;
    int len;
};

chan (Readmsg) filesys;

int read(int fd, void *data, int len)
{
    /* Pack message parameters and send to file system */
    filesys <-= (Readmsg)(fd, data, len);
}
\end{lstlisting}

If the left side of an assignment is a tuple, selected members may be
discarded by placing nil in the corresponding position in the tuple
list. In the following example only the first and third integers
returned from func are assigned.

\begin{lstlisting}
(int, int, int) func();

void main()
{
    int a, c;
    (a, nil, c) = func();
}
\end{lstlisting}

The \passthrough{\lstinline!<-=!} (assign send) operator sends the value
of the right side into a channel. The unary-expression must be of type
\passthrough{\lstinline!channel of T!}. If the left side of the
expression is of type \passthrough{\lstinline!channel of T!}, the value
transmitted down the channel is the same as if the expression were
\passthrough{\lstinline!object of type T = expression!}.

\hypertarget{promotion}{%
\subsubsection{Promotion}\label{promotion}}

If the two sides of an assignment yield different complex types then
assignment promotion is performed. The type of the right hand side is
searched for an unnamed complex type under the same rules as the
\passthrough{\lstinline!.!} operator. If a matching type is found it is
assigned to the left side. This promotion is also performed for function
arguments.

\hypertarget{polymorphic-assignment}{%
\subsubsection{Polymorphic assignment}\label{polymorphic-assignment}}

There are two operators for assigning polymorphic values. The reference
assignment operator \passthrough{\lstinline!=!} copies the fat pointer.
For example:

\begin{lstlisting}
typedef Poly;
Poly a, b;
int i;
a = (alloc Poly)i;
b = a; 
\end{lstlisting}

causes \passthrough{\lstinline!a!} to be given a fat pointer to a copy
of the variable \passthrough{\lstinline!i!} and
\passthrough{\lstinline!b!} to have a distinct fat pointer pointing to
the same copy of \passthrough{\lstinline!i!}. Polymorphic variables
assigned with the \passthrough{\lstinline!=!} operator must be of the
same polymorphic name. The value assignment operator
\passthrough{\lstinline!:=!} copies the value of one polymorphic
variable to another. The variable and value must be of the same
polymorphic name and must represent values of the same type; there is no
implicit type promotion. In particular, the variable being assigned to
must already be defined, as it must have both a type and storage. For
example:

\begin{lstlisting}
typedef Poly;
Poly a, b, c;
int i, j;
a = (alloc Poly)i;
b = (alloc Poly)j;
b := a; 
c := a; /* illegal */
\end{lstlisting}

causes \passthrough{\lstinline!a!} to be given a fat pointer to a copy
of the variable \passthrough{\lstinline!i!} and
\passthrough{\lstinline!b!} to be given a fat pointer to a copy of the
variable \passthrough{\lstinline!j!}. The value assignment
\passthrough{\lstinline!b:=a!} copies the value of
\passthrough{\lstinline!i!} from the storage referenced by the fat
pointer of \passthrough{\lstinline!a!} to the storage referenced by
\passthrough{\lstinline!b!}, with the result being that
\passthrough{\lstinline!a!} and \passthrough{\lstinline!b!} point to
distinct copies of the value of \passthrough{\lstinline!i!}; the
reference to the value of \passthrough{\lstinline!j!} is lost. The
assignment \passthrough{\lstinline!c := a!} is illegal because
\passthrough{\lstinline!c!} has no storage to hold the value;
\passthrough{\lstinline!c!} is in effect an uninitialized pointer. A
polymorphic variable may be assigned the value
\passthrough{\lstinline!nil!}. This assigns the value
\passthrough{\lstinline!0!} to the pointer element of the fat pointer
but leaves the type field unmodified.

\hypertarget{iterators}{%
\subsection{Iterators}\label{iterators}}

The iteration operator causes repeated execution of the statement that
contains the iterating expression by constructing a loop surrounding
that statement.

The operands of the iteration operator are the integral bounds of the
loop. The iteration counter may be made explicit by assigning the value
of the iteration expression to an integral variable; otherwise it is
implicit. The two expressions are evaluated before iteration begins. The
iteration is performed while the iteration counter is less than the
value of the second expression (the same convention as array bounds).
When the counter is explicit, its value is available throughout the
statement. For example, here are two implementations of a string copy
function:

\begin{lstlisting}
void copy(byte *to, byte *from)
{
    to[0::strlen(from)+1] = *from++;
}

void copy(byte *to, byte *from)
{
    int i;
    to[i] = from[i=0::strlen(from)+1];
}
\end{lstlisting}

If iterators are nested, the order of iteration is undefined.

\hypertarget{associtivity-and-precedence-of-operators}{%
\section{Associtivity and precedence of
operators}\label{associtivity-and-precedence-of-operators}}

\begin{longtable}[]{@{}ccc@{}}
\toprule
Precedence & Assoc. & Operator \\
\midrule
\endhead
14 & L to R & () {[}{]} -\textgreater{} \\
13 & R to L & ! \textasciitilde{} ++ -- \textless- ? + - * \& (cast)
sizeof zerox \\
12 & L to R & * / \% \\
11 & L to R & + - \\
10 & L to R & \textless\textless{} \textgreater\textgreater{} \\
9 & R to L & :: \\
8 & L to R & \textless{} \textless= \textgreater{} \textgreater= \\
7 & L to R & == != \\
6 & L to R & \& \\
5 & L to R & \^{} \\
4 & L to R & \textbar{} \\
3 & L to R & \&\& \\
2 & L to R & \textbar\textbar{} \\
1 & L to R & \textless-= = := += -= *= /= \%= \&= \^{}= \textbar=
\textless\textless= \textgreater\textgreater= \\
\bottomrule
\end{longtable}

% !TEX root = ../main.tex

\hypertarget{statements}{%
\chapter{Statements}\label{statements}}

Statements do not yield values, but have side effects.

\begin{lstlisting}
Statement = [ Expression ] ";" | LabelStmn | Block | SelectionStmn | LoopStmn | JumpStmn | ExceptionStmn | ProcessStmn | AllocationStmn . 
\end{lstlisting}

\hypertarget{label-statements}{%
\section{Label statements}\label{label-statements}}

\begin{lstlisting}
LabelStmn = Identifier ":" Statement . 
\end{lstlisting}

\hypertarget{blocks}{%
\section{Blocks}\label{blocks}}

\begin{lstlisting}
Block = [ "!" ] "{" [ { AutomaticDeclarations } ] | { Statement } "}" .
AutomaticDeclarations = Type [ PtrSpec ] ( FuncPtrDeclarator FuncPtrVarDecl  |  Identifier VarDecl) 
\end{lstlisting}

\hypertarget{selection}{%
\section{Selection}\label{selection}}

\begin{lstlisting}
SelectionStmn = IfElseStmn | SwitchStmn | TypeofStmn | AltStmn .
IfElseStmn    = "if" "(" Expression ")" Statement [ "else" Statement ] .
SwitchStmn    = "switch" Expression SwitchBody .
SwitchBody    = ["!"] "{" { SwitchCase } "}" .
SwitchCase    = "case" Expression ":" { Statement } | "default" ":" { Statement } .
TypeofStmn    = "typeof" Expression TypeofBody .
TypeofBody    = ["!"] "{" { TypeofCase } "}" .
TypeofCase    = "case" CastExpression ":" { Statement } | "default" ":" { Statement } .
AltStmn       = "alt" SwitchBody .
\end{lstlisting}

\hypertarget{loops}{%
\section{Loops}\label{loops}}

\begin{lstlisting}
LoopStmn  = WhileStmn | DoStmn | ForStmn .
WhileStmn = "while" "(" Expression ")" Statement .
DoStmn    = "do" Statement "while" "(" Expression ")" .
ForStmn   = "for" "(" [ Expression ] ";" [ Expression ] ";" [ Expression ] ")" Statement .
\end{lstlisting}

\hypertarget{jumps}{%
\section{Jumps}\label{jumps}}

\begin{lstlisting}
JumpStmn     = GotoStmn | ContinueStmn | BreakStmn | ReturnStmn | BecomeStmn .
GotoStmn     = "goto" Identifier ";" .
ContinueStmn = "continue" [ IntLit ] ";" .
BreakStmn    = "break" [ IntLit ] ";" .
ReturnStmn   = "return" [ Expression ] ";" .
BecomeStmn   = "become" Expression ";" .
\end{lstlisting}

\hypertarget{exceptions}{%
\section{Exceptions}\label{exceptions}}

\begin{lstlisting}
ExceptionStmn = RaiseStmn | RescueStmn | CheckStmn .
RaiseStmn     = "raise" [ Identifier ] ";" .
RescueStmn    = "rescue" ( Statement | Identifier Block ) .
CheckStmn     = "check" Expression [ "," StringLit ] ";" .
\end{lstlisting}

\hypertarget{process-control}{%
\section{Process control}\label{process-control}}

\begin{lstlisting}
ProcessStmn = ProcStmn | TaskStmn | ParStmn .
ProcStmn    = "proc" ExpressionList ";" .
TaskStmn    = "task" ExpressionList ";" .
ParStmn     = "par" Block ";" .
\end{lstlisting}

\hypertarget{allocations}{%
\section{Allocations}\label{allocations}}

\begin{lstlisting}
AllocationStmn = AllocStmn | UnallocStmn .
AllocStmn      = "alloc" ExpressionList  ";" .
UnallocStmn    = "unalloc" ExpressionList ";" .
\end{lstlisting}



%%%%%%%%%%

\appendix
% if necessary, insert here your appendices
% if necessary, insert here your bibliography

\end{document}
