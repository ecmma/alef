% !TEX root = ../main.tex

\hypertarget{type-system}{%
\chapter{Type system}\label{type-system}}

\hypertarget{basic-types}{%
\section{Basic types}\label{basic-types}}

Alef defines a small set of basic types:

\begin{table}[h]
  \centering
\begin{tabular}{|c|c|c|}
\hline
name  & size    & type                  \\ \hline
byte  & 8 bits  & unsigned byte         \\ \hline
sint  & 16 bits & signed short int      \\ \hline
usint & 16 bits & unsigned short int    \\ \hline
int   & 32 bits & signed integer        \\ \hline
uint  & 32 bits & unsigned integer      \\ \hline
float & 64 bits & floating point        \\ \hline
lint  & 64 bits & long signed integer   \\ \hline
ulint & 64 bits & long unsigned integer \\ \hline
chan  & 32 bits & channel               \\ \hline
poly  & 64 bits & polymorphic type      \\ \hline
\end{tabular}
\end{table}

where the given size is the minimum number of bits required to represent
that type. The \passthrough{\lstinline!float!} type should be the
highest precision floating point provided by the hardware, therefore
format and precision are implementation dependent. The alignment of
these types is implementation dependent as well. The
\passthrough{\lstinline!chan!} type is actually a pointer to a
runtime-system-defined object, and must be allocated before use, and are
the size of a pointer. Polymorphic types are represented by a pointer
and an hash of the type it currently represents. For a given
implementation the polymorphic type has the same size as the following
aggregate definition:

\begin{lstlisting}
aggr Polytype 
{
  void* ptr;
  int   hash;
};
\end{lstlisting}

The void type performs the special task of declaring procedures
returning no value and as part of a derived type to form generic
pointers. The void type may not be used as a basic type. The integral
types are \passthrough{\lstinline!int!}, \passthrough{\lstinline!uint!}
, \passthrough{\lstinline!sint!}, \passthrough{\lstinline!usint!} ,
\passthrough{\lstinline!byte!}, \passthrough{\lstinline!lint!} and
\passthrough{\lstinline!ulint!} . The arithmetic types are the integral
types and the type \passthrough{\lstinline!float!}. The pointer type is
a type derived from the \passthrough{\lstinline!\&!} (address of)
operator or derived from a pointer declaration.

From the point of view of the lexer types are all identifiers; however, the compiler 
needs to distinguish identifiers and typenames; therefore, for the grammars, 
identifiers naming types will be tokens of type \passthrough{\lstinline!Typename!}.
\begin{lstlisting}
BaseType = Typename [ ( ChanSpec | GenericInstantiation ) ] .
GenericInstantiation = "[" Variant "]" . 
Variant = TypeCast { "," TypeCast } . 
ChanSpec = "(" Variant ")" [ ChanBufDim ] .  
ChanBufDim = "[" Expression "]" . 
\end{lstlisting}

\hypertarget{channel-types}{%
\subsection{Channel types}\label{channel-types}}

The type specified by a chan declaration is actually a pointer to an
internal object with an anonymous type specifier. Because of their
anonymity, objects of this special type cannot be defined in
declarations; instead they must be created by an alloc statement
referring to a chan.

\hypertarget{sync-channels}{%
\subsubsection{Sync channels}\label{sync-channels}}

A channel declaration without a buffer specification produces a
synchronous communication channel. Threads sending values on the channel
will block until some other thread receives from the channel. The two
threads rendezvous and a value is passed between sender and receiver.

\hypertarget{async-channels}{%
\subsubsection{Async channels}\label{async-channels}}

If buffers are specified, then an asynchronous channel is produced. A
send operation will complete immediately while buffers are available,
and will block if all buffers are in use. A receive operation will block
if no value is buffered. If a value is buffered, the receive will
complete and make the buffer available for a new send operation. Any
senders waiting for buffers will then be allowed to continue.

\hypertarget{variant-channels}{%
\subsubsection{Variant channels}\label{variant-channels}}

If multiple types are specified in a channel definition, the channel
supplies a variant protocol. A variant protocol allows messages to be
demultiplexed by type during a receive operation. A form of the alt
statement allows the control flow to be modified based on the type of a
value received from a channel supplying a variant protocol.

\hypertarget{polymorphic-types}{%
\subsection{Polymorphic types}\label{polymorphic-types}}

The polymorphic type can be used to dynamically represent a value of any
type. A polymorphic type is identified by an identifier defined in a
polymorphic type definition or as a parameter to a polymorphic abstract
data type. Distinct identifiers represent a value of the same structure
but are different for the purposes of type checking. A polymorphic value
is represented by a fat pointer, which consists of an integer tag (an
identifier of the type, such as an hash) and a pointer to a value. Like
channels, storage for the data must be allocated by the runtime.

\hypertarget{enumerations}{%
\subsection{Enumerations}\label{enumerations}}

Enumerations are types whose value is limited to a set of integer
constants. These constants, the members of the enumeration, are called
enumerators. Enumeration variables are equivalent to integer variables.
Enumerators may appear wherever an integer constant is legal. If the
values of the enumerators are not defined explicitly, the compiler
assigns incrementing values starting from 0. If a value is given to an
enumerator, values are assigned to the following enumerators by
incrementing the value for each successive member until the next
assigned value is reached.

\hypertarget{derived-types}{%
\section{Derived types}\label{derived-types}}

Types are derived in the same way as in C. Operators applied in
declarations use one of the basic types to derive a new type. The
deriving operators are:

\begin{lstlisting}
*     create a pointer to 
&     yield the address of
()    a function returning 
[]    an array of
\end{lstlisting}

These operators bind to the name of each identifier in a declaration or
definition. Both the \passthrough{\lstinline!\&!} prefix operator and
the \passthrough{\lstinline!()!} operator have distinct rules in prefix
unary expressions and basic declarations, respectively.

\begin{lstlisting}
ArraySpec = "[" [ Expression ] "]" { "[" [ Expression ] "]" } .
PtrSpec = "*" { "*" } . 
\end{lstlisting}

\hypertarget{array-specifiers}{%
\subsection{Array Specifiers}\label{array-specifiers}}

The dimension of an array must be non-zero positive constant. Arrays
have a lower bound of $0$ and an upper bound of $n-1$, where $n$ is the value
of the constant expression.

\hypertarget{complex-types}{%
\section{Complex types}\label{complex-types}}

Complex types may be either aggregates, unions, tuples, or abstract data
types. These complex types contain sequences of basic types, derived
types and other complex types, called members. Members are referenced by
tag or by type, and members without tags are called unnamed. Arithmetic
types, channel types, tuples, and complex types may be unnamed. Derived
types may not be left unnamed. Complex unnamed members are referenced by
type or by implicit promotion during assignment or when supplied as
function arguments. Other unnamed members allocate storage but may not
be referenced. Complex types are compared by structural rather than name
equivalence.

An aggregate is a simple collection of basic, derived\footnote{\textbf{@TODO}:
  Can a function be defined inside an aggregate?} and complex types with
unique storage for each member. Unions, instead, store each member in
the same storage and its size is determined by the size of the largest
member. Abstract data types, adt's, are comparable to aggregates, but
also has a set of functions to manipulate objects of its type and a set
of visibility attributes for his members, to allow information hiding.

The declaration of complex types bind identifiers to such types, and
after declaration such identifier can be used wherever a basic type is
permitted. New type bindings may be defined from derived and basic types
using the \passthrough{\lstinline!typedef!} statement.

The complex types are \passthrough{\lstinline!aggr!},
\passthrough{\lstinline!adt!}, \passthrough{\lstinline!union!} and
\passthrough{\lstinline!tuple!}\footnote{\textbf{@TODO}: Maybe we could
  clarify, in a ``mechanical'' way, how complex type fit in the type
  system?}\footnote{\textbf{@TODO}: Clarify why tuples are different
  from other complex types, i.e.~why they can be declared as basic types
  can (inline).}.

\begin{lstlisting}
Type = BaseType | [ "tuple" ] "(" TupleList ")" .
TupleList = TypeCast "," TypeCast { "," TypeCast } .
\end{lstlisting}

What follows is a in-depth analysis of the meaning of the complex types.
Grammar rules for deriving such constructs will be given in the
Declarations section.

\hypertarget{tuples}{%
\subsection{Tuples}\label{tuples}}

A tuple is a collection of types forming a single object which can be
used in the place of an unnamed complex type. The individual members of
a tuple can only be accessed by assignment. When the declaration of a
tuple would be ambiguous because of the parenthesis (for instance in the
declaration of an automatic variable), use the keyword tuple:

\begin{lstlisting}
void
f()
{
    int a;
    tuple (int, byte, Rectangle) b;
      int c;
}
\end{lstlisting}

Type checking of tuple expressions is performed by matching the shape of
each of the component types. Tuples may only be addressed by assignment
into other complex types or l-valued tuple expressions. A parenthesized
list of expressions forms a tuple constructor, while a list of l-valued
expressions on the left hand side forms a destructor.

For example, to make a function return multiple values:

\begin{lstlisting}
(int, byte*, byte) func()
{
      return (10, "hello", 'c');
}
void main()
{
      int a;
      byte* str;
      byte c;
      (a, str, c) = func();
}
\end{lstlisting}

When a tuple appears as the left-hand side of an assignment, type
checking proceeds as if each individual member of the tuple were an
assignment statement to the corresponding member of the complex type on
the right-hand side. If a tuple appears on the right hand side of an
assignment where the left-hand side yields a complex type then the types
of each individual member of the tuple must match the corresponding
types of the complex type exactly. If a tuple is cast into a complex
type then each member of the tuple will be converted into the type of
the corresponding member of the complex type under the rules of
assignment:

\begin{lstlisting}
aggr X
{
      int a;
      byte b;
};

void main()
{
      X x; 
      byte c;
      x = (10, c);      /* Members match exactly */
      x = (X)(10, 1.5); /* float is converted to byte */
}
\end{lstlisting}

\hypertarget{abstract-data-types}{%
\subsection{Abstract data types}\label{abstract-data-types}}

An abstract data type (ADT) defines both storage for members, as in an
aggregate, and the operations that can be performed on that type. Access
to the members of an abstract data type is restricted to enable
information hiding. The scope of the members of an abstract data type
depends on their type. By default access to members that define data is
limited to the member functions. Members can be explicitly exported from
the type using the extern storage class in the member declaration.
Member functions are visible by default, the opposite behavior of data
members. Access to a member function may be restricted to other member
functions by qualifying the declaration with the intern storage class.
The four combinations are:

\begin{lstlisting}
adt Point
{
    int x;                    /* Access only by member fns */
    extern int y;             /* by everybody */
    Point set(Point*);        /* by everybody */
    intern Point tst(Point);  /* only by member fns */
};
\end{lstlisting}

\hypertarget{member-functions-methods}{%
\subsubsection{Member functions
(methods)}\label{member-functions-methods}}

Member functions are defined by type and name. The pair forms a unique
name for the function, so the same member function name can be used in
many types. Using the last example, the member function set could be
defined as:

\begin{lstlisting}
Point Point.set(Point *a)
{
    a->x = 0; /* Set Point value to zero */
    a->y = 0;
    return *a;
}
\end{lstlisting}

An implicit argument of either a pointer to the ADT or the value of the
ADT may be passed to a member function. If the first argument of the
member function declaration in the ADT specification is
\passthrough{\lstinline!* Identifier!} (with the * preceding the name of
the ADT), then a pointer to the ADT is automatically passed as the first
parameter, similarly to the \passthrough{\lstinline!self!} construct in
Smalltalk. If the declaration is of the form
\passthrough{\lstinline!. Identifier!} then the value of the ADT will be
passed to the member function, rather than a pointer to it:

\begin{lstlisting}
adt Point
{
    int x;
    extern int y;
    Point set(*Point); /* Pass &Point as 1st arg */
    Point clr(.Point); /* Pass Point as 1st arg */
    intern Point tst(Point);
};

void func()
{
    Point p;
    p.set(); /* Set receives &p as 1st arg */
}
\end{lstlisting}

\hypertarget{polymorphic-and-parametrized-adts}{%
\subsubsection{Polymorphic and parametrized
ADTs}\label{polymorphic-and-parametrized-adts}}

Alef allows the construction of type-parameterized abstract data types,
similar to generic abstract data types in Ada and Eiffel. An ADT is
parameterized by supplying type parameter names in the declaration. The
type parameters may be used to specify the types of members of the ADT.
The argument type names have the same effect as a typedef to the
polymorphic type. The scope of the types supplied as arguments is the
same as the ADT typename and can therefore be used as a type specifier
in simple declarations. For example the definition of a stack type of
parameter type \passthrough{\lstinline!T!} may be defined as:

\begin{lstlisting}
adt Stack[T]
{
    int tos;
    T data[100];
    void push(*Stack, T);
    T pop(*Stack);
};
\end{lstlisting}

Member functions of Stack are written in terms of the parameter type T.
The implementation of push might be:

\begin{lstlisting}
void Stack.push(Stack *s, T v)
{
    s->data[s->tos++] = v;
}
\end{lstlisting}

\hypertarget{bound-and-unbound-parametrized-adts}{%
\paragraph{Bound and unbound parametrized
ADTs}\label{bound-and-unbound-parametrized-adts}}

The \passthrough{\lstinline!Stack!} type can be instantiated in two
forms. In the bound form, a type is specified for T. The program is type
checked as if the supplied type were substituted for T in the ADT
declaration. For example:

\begin{lstlisting}
Stack[int] stack;
\end{lstlisting}

declares a stack where each element is an int. In the bound form a type
must be supplied for each parameter type.

In the unbound form no parameter types are specified. This allows values
of any type to be stored in the stack. For example:

\begin{lstlisting}
Stack poly;
\end{lstlisting}

declares a stack where each element has polymorphic type.

\hypertarget{conversions-and-promotions}{%
\section{Conversions and Promotions}\label{conversions-and-promotions}}

Alef performs the same implicit conversions and promotions as ANSI C
with the addition of complex type promotion: under assignment, function
parameter evaluation, or function returns, Alef will promote an unnamed
member of a complex type into the type of the left-hand side, formal
parameter, or function.

