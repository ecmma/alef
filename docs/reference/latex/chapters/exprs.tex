% !Tex root = ../main.tex
\hypertarget{expressions}{%
\chapter{Expressions}\label{expressions}}

The order of expression evaluation is not defined except where noted.
That is, unless the definition of the operator guarantees evaluation
order, an operator may evaluate any of its operands first. The behavior
of exceptional conditions such as divide by zero, arithmetic overflow,
and floating point exceptions is not defined by the specification and is
implementation dependent.

\hypertarget{pointer-generation}{%
\section{Pointer generation}\label{pointer-generation}}

References to expressions of type
\passthrough{\lstinline!function returning T!} and
\passthrough{\lstinline!array of T!} are rewritten to produce pointers
to either the function or the first element of the array. That is,
\passthrough{\lstinline!function returning T!} becomes
\passthrough{\lstinline!pointer to function returning T!} and
\passthrough{\lstinline!array of T!} becomes
\passthrough{\lstinline!pointer to the first element of array of T!}.

\hypertarget{primary-expressions}{%
\section{Primary expressions}\label{primary-expressions}}

Primary expressions are identifiers, constants, or parenthesized
expressions:

\begin{lstlisting}
PrimaryExpression = Identifier | Literal | "nil" | [ "tuple" ] "(" ExpressionList ")" .
\end{lstlisting}

The primary expression \passthrough{\lstinline!nil!} returns a pointer
of type \passthrough{\lstinline!pointer to void!} of value 0 which is
guaranteed not to point at an object. \passthrough{\lstinline!nil!} may
also be used to initialize channels and polymorphic types to a known
value. The only legal operation on these types after such an assignment
is a test with one of the equality test operators and the
\passthrough{\lstinline!nil!} value.

\hypertarget{postfix-expressions}{%
\section{Postfix expressions}\label{postfix-expressions}}

\begin{lstlisting}
PostfixExpression = ( PrimaryExpression | AdtNameCall ) { PostfixOperand } .
AdtNameCall       = "." Typename "." Identifier .
PostfixOperand    = ArrayAccess | FuncCall | MemberAccess | IndirectAccess | UnaryPostfix .
ArrayAccess       = "[" Expression "]" .
FuncCall          = "(" [ ExpressionList ] ")" .
MemberAccess      = "." Identifier .
IndirectAccess    = "->" Identifier .
UnaryPostfix      = "++" | "--" | "?" .
ExpressionList    = Expression { "," Expression } .
\end{lstlisting}

\hypertarget{array-reference}{%
\subsection{Array reference}\label{array-reference}}

A primary expression followed by an expression enclosed in square
brackets is an array indexing operation. The expression is rewritten to
be

\begin{lstlisting}
*((PrimaryExpression)+(Expression))
\end{lstlisting}

One of the expressions must be of type pointer, the other of integral
type.

\hypertarget{function-calls}{%
\subsection{Function calls}\label{function-calls}}

Function call postfix operators yield a value of type
\passthrough{\lstinline!pointer to function!}. A type declaration for
the function must be declared prior to a function call. The declaration
can be either the definition of the function or a function prototype.
The types of each argument in the prototype must match the corresponding
expression type under the rules of promotion and conversion for
assignment.

\hypertarget{function-call-promotions}{%
\subsubsection{Function call
promotions}\label{function-call-promotions}}

In addition, unnamed complex type members will be promoted
automatically. For example:

\begin{lstlisting}
aggr Test
{
    int t;
    Lock; /* Unnamed substructure */
};

Test yuk;   /* Definition of complex object yuk */

void lock(Lock*); /* Prototype for function lock */

void main()
{
    lock(&yuk); /* address of yuk.Lock is passed */
}
\end{lstlisting}

\hypertarget{adt-namecalls}{%
\subsubsection{ADT namecalls}\label{adt-namecalls}}

Calls to member functions may use the type name instead of an expression
to identify the ADT. If the function has an implicit first parameter,
\passthrough{\lstinline!nil!} is passed. Given the following definition
of \passthrough{\lstinline!X!} these two calls are equivalent:

\begin{lstlisting}
adt X
{
    int i;
    void f(*X);
};

X val;

((X*)nil)->f();

.X.f();
\end{lstlisting}

This form is illegal if the implicit parameter is declared by value
rather than by reference.

\hypertarget{polymorphic-promotions}{%
\subsubsection{Polymorphic promotions}\label{polymorphic-promotions}}

Calls to member functions of polymorphic ADT's have special promotion
rules for function arguments. If a polymorphic type
\passthrough{\lstinline!P!} has been bound to an actual type
\passthrough{\lstinline!T!} then an actual parameter
\passthrough{\lstinline!v!} of type \passthrough{\lstinline!T!}
corresponding to a formal parameter of type \passthrough{\lstinline!P!}
will be promoted into type \passthrough{\lstinline!P!} automatically.
The promotion is equivalent to \passthrough{\lstinline!(alloc P)v!} as
described in the Casts section. For example:

\begin{lstlisting}
adt X[T]
{
    void f(*X, T);
};

X[int] bound;

bound.f(3);           /* 3 is promoted as if (alloc T)3 */
bound.f((alloc T)3);  /* illegal: int not same as poly */
\end{lstlisting}

In the unbound case values must be explicitly converted into the
polymorphic type using the cast syntax:

\begin{lstlisting}
X unbound;

unbound.f((alloc T)3);  /* 3 is converted into poly */
unbound.f(3);           /* illegal: int not same as poly */
\end{lstlisting}

In either case, the actual parameter must have the same type as the
formal parameter after any binding has taken place.

\hypertarget{complex-type-references}{%
\subsection{Complex type references}\label{complex-type-references}}

The operator \passthrough{\lstinline!.!} references a member of a
complex type. The first part of the expression must yield
\passthrough{\lstinline!union!}, \passthrough{\lstinline!aggr!}, or
\passthrough{\lstinline!adt!}. Named members must be specified by name,
unnamed members by type. Only one unnamed member of type typename is
permitted in the complex type when referencing members by type,
otherwise the reference would be ambiguous.

If the reference is by typename and no members of typename exist in the
complex, unnamed substructures will be searched breadth first. The
operation \passthrough{\lstinline!$->$!} uses a pointer to reference a
complex type member. The \passthrough{\lstinline!$->$!} operator follows
the same search and type rules as \passthrough{\lstinline!.!} and is
equivalent to the expression
\passthrough{\lstinline!(*PostfixExpression).tag.!}

References to polymorphic members of unbound polymorphic ADT's behave as
normal members: they yield an unbound polymorphic type. Bound
polymorphic ADT's have special rules. Consider a polymorphic type
\passthrough{\lstinline!P!} that is bound to an actual type
\passthrough{\lstinline!T!}. If a reference to a member or function
return value of type \passthrough{\lstinline!P!} is assigned to a
variable \passthrough{\lstinline!v!} of type \passthrough{\lstinline!T!}
using the assignment operator \passthrough{\lstinline!=!}, then the type
of \passthrough{\lstinline!P!} will be narrowed to
\passthrough{\lstinline!T!}, assigned to \passthrough{\lstinline!v!},
and the storage used by the polymorphic value will be unallocated. The
value assignment operator \passthrough{\lstinline!:=!} performs the same
type narrowing but does not unallocate the storage used by the
polymorphic value. For example:

\begin{lstlisting}
adt Stack[T]
{
    int tos;
    T data[100];
};

Stack[int] s;
int i, j, k;

i := s.data[s->tos];
j = s.data[s->tos];
k = s.data[s->tos]; /* s.data[s->tos] has been unallocated. */
\end{lstlisting}

The first assignment copies the value at the top of the stack into
\passthrough{\lstinline!i!} without altering the data structure. The
second assignment moves the value into \passthrough{\lstinline!j!} and
unallocates the storage used in the stack data structure. The third
assignment is illegal since \passthrough{\lstinline!data[s->tos]!} has
been unallocated.

\hypertarget{postfix-increment-and-decrement}{%
\subsection{Postfix increment and
decrement}\label{postfix-increment-and-decrement}}

The postfix increment ( \passthrough{\lstinline!++!} ) and decrement (
\passthrough{\lstinline!--!} ) operators return the value of expression,
then increment it or decrement it by 1. The expression must be an
l-value of integral or pointer type.

\hypertarget{prefix-expressions}{%
\section{Prefix expressions}\label{prefix-expressions}}

\begin{lstlisting}
UnaryExpression = PostfixExpression | UnaryPrefix | CastPrefix .
UnaryPrefix     = ( "<-" | "++" | "--" | "zerox" ) UnaryExpression .
CastPrefix      = UnaryOperator Term .
UnaryOperator   = ( "?" | "*" | "&" | "!" | "+" | "-" | "~" | "sizeof" ) .
\end{lstlisting}

\hypertarget{prefix-increment-and-decrement}{%
\subsection{Prefix increment and
decrement}\label{prefix-increment-and-decrement}}

The prefix increment ( \passthrough{\lstinline!++!} ) and prefix
decrement ( \passthrough{\lstinline!--!} ) operators add or subtract one
to a unary expression and return the new value. The unary expression
must be an l-value of integral or pointer type.

\hypertarget{receive-and-can-receive}{%
\subsection{Receive and can receive}\label{receive-and-can-receive}}

The prefix operator \passthrough{\lstinline!<-!} receives a value from a
channel. The unary expression must be of type
\passthrough{\lstinline!channel of T!}. The type of the result will be
\passthrough{\lstinline!T!}. A process or task will block until a value
is available from the channel. The \emph{prefix} operator
\passthrough{\lstinline!?!} returns \passthrough{\lstinline!1!} if a
channel has a value available for receive, \passthrough{\lstinline!0!}
otherwise.

\hypertarget{send-and-can-send}{%
\subsection{Send and Can send}\label{send-and-can-send}}

The postfix operator \passthrough{\lstinline!<-!}, on the left-hand side
of an assignment sends a value to a channel, for example:

\begin{lstlisting}
chan(int) c;
c <-= 1;    /* send 1 on channel c */
\end{lstlisting}

The \emph{postfix} operator \passthrough{\lstinline!?!} returns
\passthrough{\lstinline!1!} if a thread can send on a channel without
blocking, \passthrough{\lstinline!0!} otherwise. The prefix or postfix
blocking test operator ? is only reliable when used on a channel shared
between tasks in a single process. A process may block after a
successful \passthrough{\lstinline!?!} because there may be a race
between processes competing for the same channel.

\hypertarget{indirection}{%
\subsection{Indirection}\label{indirection}}

The unary prefix operator \passthrough{\lstinline!*!} retrieves the
value pointed to by its operand. The operand must be of type
\passthrough{\lstinline!pointer to T!}. The result of the indirection is
a value of type \passthrough{\lstinline!T!}.

\hypertarget{unary-plus-and-minus}{%
\subsection{Unary plus and minus}\label{unary-plus-and-minus}}

Unary plus is equivalent to
\passthrough{\lstinline!(0+(UnaryExpression))!}. Unary minus is
equivalent to \passthrough{\lstinline!(0-(UnaryExpression))!}. An
integral operand undergoes integral promotion. The result has the type
of the promoted operand.

\hypertarget{bitwise-negate}{%
\subsection{Bitwise negate}\label{bitwise-negate}}

The operator \passthrough{\lstinline!\~!} performs a bitwise negation of
its operand, which must be of integral type.

\hypertarget{logical-negate}{%
\subsection{Logical negate}\label{logical-negate}}

The operator \passthrough{\lstinline"!"} performs logical negation of
its operand, which must of arithmetic or pointer type. If the operand is
a pointer and its value is \passthrough{\lstinline!nil!} the result is
integer \passthrough{\lstinline!1!}, otherwise
\passthrough{\lstinline!0!}. If the operand is arithmetic and the value
is \passthrough{\lstinline!0!} the result is
\passthrough{\lstinline!1!}, otherwise the result is
\passthrough{\lstinline!0!}.

\hypertarget{zerox}{%
\subsection{Zerox}\label{zerox}}

The \passthrough{\lstinline!zerox!} operator may only be applied to an
expression of polymorphic type. The result of
\passthrough{\lstinline!zerox!} is a new fat pointer, which points at a
copy of the result of evaluating its operand. For example:

\begin{lstlisting}
typedef Poly;
Poly a, b, c;
a = (alloc Poly)10;
b = a; 
c = zerox a;
\end{lstlisting}

causes \passthrough{\lstinline!a!} and \passthrough{\lstinline!b!} to
point to the same storage for the value \passthrough{\lstinline!10!} and
\passthrough{\lstinline!c!} to point to distinct storage containing
another copy of the value \passthrough{\lstinline!10!}.

\hypertarget{sizeof-operator}{%
\subsection{Sizeof operator}\label{sizeof-operator}}

The \passthrough{\lstinline!sizeof!} operator yields the size in
\passthrough{\lstinline!byte!}s of its operand, which may be an
expression or the parenthesized name of a type. The size is determined
from the type of the operand, which is not itself evaluated. The result
is a \passthrough{\lstinline!signed integer!} constant.

\hypertarget{term-expressions}{%
\section{Term expressions}\label{term-expressions}}

\begin{lstlisting}
Term            = UnaryExpression | CastExpression | AllocExpression .
CastExpression  = "(" TypeCast ")" Term .
AllocExpression = "(" "alloc" Typename ")" Term .
TypeCast        = BaseType [ PtrSpec ] [ FuncCast ] | "tuple" "(" TupleList ")" .
FuncCast        = "(" [ PtrSpec ] ")" "(" [ ParamList ] ")" .
\end{lstlisting}

\hypertarget{cast-expressions}{%
\subsection{Cast expressions}\label{cast-expressions}}

A cast converts the result of an expression into a new type. A value of
any type may be converted into a polymorphic type by adding the keyword
\passthrough{\lstinline!alloc!} before the polymorphic type name. This
has the effect of allocating storage for the value, assigning the value
of the operand into the storage, and yielding a fat pointer as the
result. For example, to create a polymorphic variable with integer value
\passthrough{\lstinline!10!}:

\begin{lstlisting}
typedef Poly;
Poly p;
p = (alloc Poly) 10;
\end{lstlisting}

The only other legal cast involving a polymorphic type converts one
polyname into another.

\hypertarget{binary-expressions}{%
\section{Binary expressions}\label{binary-expressions}}

Binary operators in LL grammars lose their left associativity. A given
implementation will use Dijkstra's shunting yard algorithm or a Pratt
Parser. Nonetheless, a list of valid expressions follows.

\begin{lstlisting}
Expression   = Term | Expression BinaryOp Expression .
BinaryOp     = SumOp | MulOp | LogOp | ShOp | CompOp | EqOp | AssOp | IterOp .
SumOp        = "+" | "-" .
MulOp        = "\*" | "/" .
LogOp        = BitwiseLogOp | "&&" | "||" .
BitwiseLogOp = "^" | "&" | "|" .
ShOp         = "<<" | ">>" .
CompOp       = "<" | ">" | ">= " | "<=" .
EqOp         = "==" | "!=" . 
AssOp        = "=" | ":=" | "<-=" | (SumOp | MulOp | BitwiseLogOp | ShOp ) "=" .
IterOp       = "::" .
\end{lstlisting}

\hypertarget{multiply-divide-and-modulus}{%
\subsection{Multiply, divide and
modulus}\label{multiply-divide-and-modulus}}

The operands of \passthrough{\lstinline!*!} and
\passthrough{\lstinline!/!} must have arithmetic type. The operands of
\passthrough{\lstinline!\%!} must be of integral type. The operator
\passthrough{\lstinline!/!} yields the quotient,
\passthrough{\lstinline!\%!} the remainder, and
\passthrough{\lstinline!*!} the product of the operands. If
\passthrough{\lstinline!b!} is non-zero then
\passthrough{\lstinline!a == (a/b) + a\%b!} should always be true.

\hypertarget{add-and-subtract}{%
\subsection{Add and subtract}\label{add-and-subtract}}

The \passthrough{\lstinline!+!} operator computes the sum of its
operands. Either one of the operands may be a pointer. If
\passthrough{\lstinline!P!} is an expression yielding a pointer to type
\passthrough{\lstinline!T!} then \passthrough{\lstinline!P+n!} is the
same as \passthrough{\lstinline!p+(sizeof(T)*n)!}. The
\passthrough{\lstinline!-!} operator computes the difference of its
operands. The first operand may be of pointer or arithmetic type. The
second operand must be of arithmetic type. If
\passthrough{\lstinline!P!} is an expression yielding a pointer of type
\passthrough{\lstinline!T!} then \passthrough{\lstinline!P-n!} is the
same as \passthrough{\lstinline!p-(sizeof(T)*n)!}. Thus if
\passthrough{\lstinline!P!} is a pointer to an element of an array,
\passthrough{\lstinline!P+1!} will point to the next object in the array
and \passthrough{\lstinline!P-1!} will point to the previous object in
the array.

\hypertarget{shift-operators}{%
\subsection{Shift operators}\label{shift-operators}}

The shift operators perform bitwise shifts. If the first operand is
unsigned, \passthrough{\lstinline!<<!} performs a logical left shift by
a number of bits as its right operand. If the first operand is signed,
\passthrough{\lstinline!<<!} performs an arithmetic left shift by a
number of bits as its right operand. The left operand must be of
integral type. The \passthrough{\lstinline!>>!} operator is a right
shift and follows the same rules as left shift.

\hypertarget{relational-operators}{%
\subsection{Relational Operators}\label{relational-operators}}

The values of expressions can be compared using relational operators.
The operators are \passthrough{\lstinline!<!} (less than),
\passthrough{\lstinline!>!} (greater than), \passthrough{\lstinline!<=!}
(less than or equal to) and \passthrough{\lstinline!>=!} (greater than
or equal to). The operands must be of arithmetic or pointer type. The
value of the expression is \passthrough{\lstinline!1!} if the relation
is true, otherwise \passthrough{\lstinline!0!}. The usual arithmetic
conversions are performed. Pointers may only be compared to pointers of
the same type or of type \passthrough{\lstinline!void*!}.

\hypertarget{equality-operators}{%
\subsection{Equality operators}\label{equality-operators}}

The operators \passthrough{\lstinline!==!} (equal to) and
\passthrough{\lstinline"!="} (not equal) follow the same rules as
relational operators. The equality operations may be applied to
expressions yielding channels and polymorphic types for comparison with
the value \passthrough{\lstinline!nil!}. A pointer of value
\passthrough{\lstinline!nil!} or type \passthrough{\lstinline!void*!}
may be compared to any pointer.

\hypertarget{bitwise-logic-operators}{%
\subsection{Bitwise logic operators}\label{bitwise-logic-operators}}

The bitwise logic operators perform bitwise logical operations and apply
only to integral types. The operators are \passthrough{\lstinline!\&!}
(bitwise and), \passthrough{\lstinline!\^!} (bitwise exclusive or) and
\passthrough{\lstinline!|!} (bitwise inclusive or).

\hypertarget{logical-operators}{%
\subsection{Logical operators}\label{logical-operators}}

The \passthrough{\lstinline!\&\&!} operator returns
\passthrough{\lstinline!1!} if both of its operands evaluate to
non-zero, otherwise \passthrough{\lstinline!0!}. The
\passthrough{\lstinline!||!} operator returns
\passthrough{\lstinline!1!} if either of its operand evaluates to
non-zero, otherwise \passthrough{\lstinline!0!}. Both operators are
guaranteed to evaluate strictly left to right. Evaluation of the
expression will cease as soon the final value is determined. The
operands can be any mix of arithmetic and pointer types.

\hypertarget{constant-expressions}{%
\subsection{Constant expressions}\label{constant-expressions}}

A constant expression is an expression which can be fully evaluated by
the compiler during translation rather than at runtime.
Constant expression appears as part of initialization, channel buffer
specifications, and array dimensions. The following operators may not be
part of a constant expression: function calls, assignment, send,
receive, increment and decrement. Address computations using the
\passthrough{\lstinline!\&!} (address of) operator on static
declarations is permitted.

\hypertarget{assignment}{%
\subsection{Assignment}\label{assignment}}

The assignment operators are:

\begin{lstlisting}
= := += *= /= -= %= &= |= ^= >>= <<= 
\end{lstlisting}

The left side of the expression must be an l-value. Compound assignment
allows the members of a complex type to be assigned from a member list
in a single statement. A compound assignment is formed by casting a
tuple into the complex type. Each element of the tuple is evaluated in
turn and assigned to its corresponding element in the complex types. The
usual conversions are performed for each assignment.

\begin{lstlisting}
/* Encoding of read message to send to file system */
aggr Readmsg
{
    int fd;
    void *data;
    int len;
};

chan (Readmsg) filesys;

int read(int fd, void *data, int len)
{
    /* Pack message parameters and send to file system */
    filesys <-= (Readmsg)(fd, data, len);
}
\end{lstlisting}

If the left side of an assignment is a tuple, selected members may be
discarded by placing nil in the corresponding position in the tuple
list. In the following example only the first and third integers
returned from func are assigned.

\begin{lstlisting}
(int, int, int) func();

void main()
{
    int a, c;
    (a, nil, c) = func();
}
\end{lstlisting}

The \passthrough{\lstinline!<-=!} (assign send) operator sends the value
of the right side into a channel. The unary-expression must be of type
\passthrough{\lstinline!channel of T!}. If the left side of the
expression is of type \passthrough{\lstinline!channel of T!}, the value
transmitted down the channel is the same as if the expression were
\passthrough{\lstinline!object of type T = expression!}.

\hypertarget{promotion}{%
\subsubsection{Promotion}\label{promotion}}

If the two sides of an assignment yield different complex types then
assignment promotion is performed. The type of the right hand side is
searched for an unnamed complex type under the same rules as the
\passthrough{\lstinline!.!} operator. If a matching type is found it is
assigned to the left side. This promotion is also performed for function
arguments.

\hypertarget{polymorphic-assignment}{%
\subsubsection{Polymorphic assignment}\label{polymorphic-assignment}}

There are two operators for assigning polymorphic values. The reference
assignment operator \passthrough{\lstinline!=!} copies the fat pointer.
For example:

\begin{lstlisting}
typedef Poly;
Poly a, b;
int i;
a = (alloc Poly)i;
b = a; 
\end{lstlisting}

causes \passthrough{\lstinline!a!} to be given a fat pointer to a copy
of the variable \passthrough{\lstinline!i!} and
\passthrough{\lstinline!b!} to have a distinct fat pointer pointing to
the same copy of \passthrough{\lstinline!i!}. Polymorphic variables
assigned with the \passthrough{\lstinline!=!} operator must be of the
same polymorphic name. The value assignment operator
\passthrough{\lstinline!:=!} copies the value of one polymorphic
variable to another. The variable and value must be of the same
polymorphic name and must represent values of the same type; there is no
implicit type promotion. In particular, the variable being assigned to
must already be defined, as it must have both a type and storage. For
example:

\begin{lstlisting}
typedef Poly;
Poly a, b, c;
int i, j;
a = (alloc Poly)i;
b = (alloc Poly)j;
b := a; 
c := a; /* illegal */
\end{lstlisting}

causes \passthrough{\lstinline!a!} to be given a fat pointer to a copy
of the variable \passthrough{\lstinline!i!} and
\passthrough{\lstinline!b!} to be given a fat pointer to a copy of the
variable \passthrough{\lstinline!j!}. The value assignment
\passthrough{\lstinline!b:=a!} copies the value of
\passthrough{\lstinline!i!} from the storage referenced by the fat
pointer of \passthrough{\lstinline!a!} to the storage referenced by
\passthrough{\lstinline!b!}, with the result being that
\passthrough{\lstinline!a!} and \passthrough{\lstinline!b!} point to
distinct copies of the value of \passthrough{\lstinline!i!}; the
reference to the value of \passthrough{\lstinline!j!} is lost. The
assignment \passthrough{\lstinline!c := a!} is illegal because
\passthrough{\lstinline!c!} has no storage to hold the value;
\passthrough{\lstinline!c!} is in effect an uninitialized pointer. A
polymorphic variable may be assigned the value
\passthrough{\lstinline!nil!}. This assigns the value
\passthrough{\lstinline!0!} to the pointer element of the fat pointer
but leaves the type field unmodified.

\hypertarget{iterators}{%
\subsection{Iterators}\label{iterators}}

The iteration operator causes repeated execution of the statement that
contains the iterating expression by constructing a loop surrounding
that statement.

The operands of the iteration operator are the integral bounds of the
loop. The iteration counter may be made explicit by assigning the value
of the iteration expression to an integral variable; otherwise it is
implicit. The two expressions are evaluated before iteration begins. The
iteration is performed while the iteration counter is less than the
value of the second expression (the same convention as array bounds).
When the counter is explicit, its value is available throughout the
statement. For example, here are two implementations of a string copy
function:

\begin{lstlisting}
void copy(byte *to, byte *from)
{
    to[0::strlen(from)+1] = *from++;
}

void copy(byte *to, byte *from)
{
    int i;
    to[i] = from[i=0::strlen(from)+1];
}
\end{lstlisting}

If iterators are nested, the order of iteration is undefined.

\hypertarget{associtivity-and-precedence-of-operators}{%
\section{Associtivity and precedence of
operators}\label{associtivity-and-precedence-of-operators}}

\begin{longtable}[]{@{}ccc@{}}
\toprule
Precedence & Assoc. & Operator \\
\midrule
\endhead
14 & L to R & () {[}{]} -\textgreater{} \\
13 & R to L & ! \textasciitilde{} ++ -- \textless- ? + - * \& (cast)
sizeof zerox \\
12 & L to R & * / \% \\
11 & L to R & + - \\
10 & L to R & \textless\textless{} \textgreater\textgreater{} \\
9 & R to L & :: \\
8 & L to R & \textless{} \textless= \textgreater{} \textgreater= \\
7 & L to R & == != \\
6 & L to R & \& \\
5 & L to R & \^{} \\
4 & L to R & \textbar{} \\
3 & L to R & \&\& \\
2 & L to R & \textbar\textbar{} \\
1 & L to R & \textless-= = := += -= *= /= \%= \&= \^{}= \textbar=
\textless\textless= \textgreater\textgreater= \\
\bottomrule
\end{longtable}
