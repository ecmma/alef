% !TEX root = ../main.tex 

\hypertarget{declarations}{%
\chapter{Declarations}\label{declarations}}

\hypertarget{programs}{%
\section{Programs}\label{programs}}

A declaration attaches a type to an identifier; it need not reserve
storage. A declaration which reserves storage is called a definition. A
program consists of a list of declarations. A declaration can define a
simple variable, a function, a prototype to a function, an ADT function,
a type specification, or a type definition.

\begin{lstlisting}
Program     = { Declaration } .
Declaration = [ Visibility ] ( SimpleDecl | ComplexDecl | TypeDefs ).
Visibility  = "intern" | "extern" .
\end{lstlisting}

A declaration introduces an identifier and specifies its type. A
definition is a declaration that also reserves storage for an
identifier. An object is an area of memory of known type produced by a
definition. Function prototypes, variable declarations preceded by
extern, and type specifiers are declarations. Function declarations with
bodies and variable declarations are examples of definitions.

\hypertarget{scopes}{%
\subsection{Scopes}\label{scopes}}

Scopes define where a named object, that is an identifier with a precise
semantic meaning, can be referenced.

\hypertarget{file-scope}{%
\subsubsection{File scope}\label{file-scope}}

A declaration introduces an identifier and specifies its type. A
definition is a declaration that also reserves storage for an
identifier. Declaration at the file scope are implicitly assumed to be
``extern'' declarations, that is, visible to all compilations units. If
a declaration is preceded with the ``intern'' keyword, its scope is
narrowed to its compilation unit only. Variable declarations (that is,
variable declarations preceded by the keyword ``extern'') aren't
definitions, and therefore do not allocate space for the variable.
Similarly, function prototypes do not allocate space\footnote{\textbf{@TODO}:
  The use of extern as a keyword to make a variable visible in the
  entire compilation unit and to make non-allocating definitions is
  highly unclear -- we might use a different keyword to indicate
  non-allocating definitions.}.

\hypertarget{type-scope}{%
\subsubsection{Type scope}\label{type-scope}}

Members of complex types (ADT's, aggregates or union) are in scope only
when an access operator is applied to objects of their appartaining
type. Members of unions and aggregates are always accessible from
outside. Access to members of ADT's can be restricted (or enabled) using
visibility specifiers ``intern'' and ``extern'': variable members are,
by default, ``intern'', that is not accessible from outside. Methods
(function members of an ADT) are by default ``extern''.

\hypertarget{function-scope}{%
\subsubsection{Function scope}\label{function-scope}}

Labels and raise statement's identifiers can be referenced from the
start of a function to its end, regardless of the position of the
declaration.

\hypertarget{local-scope}{%
\subsubsection{Local scope}\label{local-scope}}

Local identifiers are declared at the start of a block\footnote{\textbf{@TODO}:
  This means that, in a function's body, automatic variables can be
  declared only at the start, before any statement.}. A local identifier
has scope starting from its declaration to the end of the block in which
it was declared.

\hypertarget{storage-classes}{%
\subsection{Storage classes}\label{storage-classes}}

While scopes define where identifiers have meaning, storage classes
define the lifetime ``of the meaning'', that is, when variables and
functions are created and deleted, and to what value they're
initialized.

\hypertarget{automatic-storage-class}{%
\subsubsection{Automatic storage class}\label{automatic-storage-class}}

Automatic objects are created at the entry of the block in which they
were declared, and their value is undefined upon creation.

\hypertarget{parameter-storage-class}{%
\subsubsection{Parameter storage class}\label{parameter-storage-class}}

Function parameters are created by function invocation and are destroyed
at function exit. They have the value of the values passed by the
caller.

\hypertarget{static-storage-class}{%
\subsubsection{Static storage class}\label{static-storage-class}}

Static objects exist from invocation of the program until termination,
and uninitialized static objects have, at creation, the value 0.

\hypertarget{simple-declarations}{%
\section{Simple declarations}\label{simple-declarations}}

A simple declaration consists of a type specifier and a list of
identifiers. Each identifier may be qualified by deriving operators.
Simple declarations at the file scope may be initialized. Function
pointer declarations have \emph{per sé} rules.

\begin{lstlisting}
SimpleDecl = Type [ PtrSpec ] ( FuncPtr | BaseDecl ) . 
FuncPtr    = "(" [ PtrSpec ] Identifier ( FuncPtrFuncDecl | FuncPtrVarDecl ) .
BaseDecl   = Identifier ( FuncDecl | VarDecl |  MethDecl ) .
\end{lstlisting}

\hypertarget{function-pointer-declarations}{%
\subsection{Function pointer
declarations}\label{function-pointer-declarations}}

\begin{lstlisting}
FuncPtrFuncDecl  = "(" [ ParamList ] ")" ")" "(" [ ParamList ] ")" ( ";" | Block ) .
ParamList         = Param { "," Param } .
Param            = SimpleParam | TupleParam  | "..." .
SimpleParam      = BaseType [ PtrSpec ] [ ( BaseParam |  FuncPtrParam ) ] . 
BaseParam        = Identifier [ ArraySpec ] . 
FuncPtrParam     = "(" PtrSpec [ Identifier ] [ ArraySpec ] ")"  "(" [ ParamList ] ")" .
TupleParam       = "tuple" "(" TupleList ")" [ [ PtrSpec ] ( BaseParam | FuncPtrParam ) ] .
FuncPtrVarDecl   = [ ArraySpec ] ")"  "(" [ ParamList ] ")" [ "=" InitExpression ] ( ";" |  "," [ PtrSpec ] ( "(" [ PtrSpec ] Identifier FuncPtrVarDecl | Identifier VarDecl ) ) .
\end{lstlisting}

The parameters received by a function taking variable arguments are
referenced using the ellipsis \passthrough{\lstinline!...!}. The token
\passthrough{\lstinline!...!} yields is a value of type
\passthrough{\lstinline!pointer to void!}. The value points at the first
location after the formal parameters.

\hypertarget{examples-of-function-pointer-declarations.}{%
\subsubsection{Examples of function pointer
declarations.}\label{examples-of-function-pointer-declarations.}}

\hypertarget{a-variable-of-type-function-pointer.}{%
\paragraph{A variable of type array of function pointer.}
\label{a-variable-of-type-function-pointer.}}
In this example,
\begin{lstlisting}
int * (* func_ptr[2] ) (int, bool);
\end{lstlisting}

\hypertarget{a-function-returning-a-function.}{%
\paragraph{A function returning a
function.}\label{a-function-returning-a-function.}}
In this example,

\begin{lstlisting}
int * (func_func (float, char)) (int, bool);
\end{lstlisting}

\hypertarget{a-notable-case}{%
\paragraph{A notable case}\label{a-notable-case}}
In this example,

\begin{lstlisting}
int * (func) (int, bool);
\end{lstlisting}

Is interpreted as a function, not as a variable of type pointer to a
function.

\hypertarget{variable-function-and-method-declarations}{%
\subsection{Variable, function and method
declarations}\label{variable-function-and-method-declarations}}

\begin{lstlisting}
FuncDecl = "(" [ ParamList ] ")" ( ";" | Block ) .
VarDecl  = [ ArraySpec ] [ "=" InitExpression ] ( ";" | "," [ PtrSpec ] ( "(" [ PtrSpec ] Identifier FuncPtrVarDecl | Identifier VarDecl ) .
MethDecl = "." Identifier "(" [ ParamList ] ")" Block .
\end{lstlisting}

\hypertarget{initializers}{%
\subsubsection{Initializers}\label{initializers}}
Only simple declarations at the file scope may be initialized\footnote{We may want to change this.}. 
An initialization consists of a constant expression or a list of
constant expressions separated by commas and enclosed by braces. An
array or complex type requires an explicit set of braces for each level
of nesting. Unions may not be initialized. All the components of a
variable need not be explicitly initialized; uninitialized elements are
set to zero. ADT types are initialized in the same way as aggregates
with the exception of ADT function members which are ignored for the
purposes of initialization. Elements of sparse arrays can be initialized
by supplying a bracketed index for an element. Successive elements
without the index notation continue to initialize the array in sequence.
For example:

\begin{lstlisting}
byte a[256] = {
    ['a'] 'A',    /* Set element 97 to 65 */
    ['a'+1] 'B',  /* Set element 98 to 66 */
    'C'           /* Set element 99 to 67 */
};
\end{lstlisting}

If the dimensions of the array are omitted from the array-spec the
compiler sets the size of each dimension to be large enough to
accommodate the initialization. The size of the array in bytes can be
found using sizeof.

\begin{lstlisting}
InitExpression     = Expression | ArrayElementInit | MemberInit | BlockInit .
ArrayElementInit   = "[" Expression "]"  ( Expression | BlockInit ) .
MemberInit         = "." Identifier Expression .
BlockInit = "{" [ InitExpressionList ] "}" .
InitExpressionList = InitExpression [ "," InitExpression ] .
\end{lstlisting}

\hypertarget{complex-type-declaration}{%
\section{Complex type declaration}\label{complex-type-declaration}}

Complex declarations define new aggregates, unions, ADT's and enums in
the innermost active scope.

\begin{lstlisting}
ComplexDecl = ( AggrDecl | UnionDecl | AdtDecl | EnumDecl ) ";" .
\end{lstlisting}

\hypertarget{unions-and-aggregates}{%
\subsection{Unions and aggregates}\label{unions-and-aggregates}}

\begin{lstlisting}
AggrDecl        = "aggr" [ Identifier ] "{" { AggrUnionMember } "}" [ Identifier ] .
UnionDecl       = "union" [ Identifier ] "{" { AggrUnionMember } "}" [ Identifier ] .
AggrUnionMember = ComplexDefs | VarMember .
VarMember       = Type [ [ PtrSpec ]  ( SimpleMember | FuncPtrMember ) { "," [ PtrSpec ] ( SimpleMember | FuncPtrMember ) } ] ";" .
SimpleMember    = Identifier [ ArraySpec ] .
FuncPtrMember   = "(" [ PtrSpec ] Identifier [ ArraySpec ] ")" "(" [ParamList] ")" .
\end{lstlisting}

\hypertarget{abstract-data-types-1}{%
\subsection{Abstract data types}\label{abstract-data-types-1}}

\begin{lstlisting}
AdtDecl                = "adt" [ Identifier ] [ AdtGenSpec ] "{" { AdtMember } "}" [ Identifier ] .
AdtGenSpec             = "[" Identifier { "," Identifier "}" "]" .
AdtMember              = [ Visibility ] Type [ [ PtrSPec ] ( AdtFuncPtrMember | AdtBaseMember ) ] ";" .
AdtFuncPtrMember       = "(" [ PtrSpec ] Identifier ( AdtFuncPtrMethodMember | AdtFuncPtrVarMember ) .
AdtFuncPtrMethodMember = "(" [ AdtMethodRefParam [ "," ParamList ] ] | ParamList ")" ")" "(" [ ParamList ] ")" .
AdtFuncPtrVarMember    = [ ArraySpec ] ")" "(" [ ParamList ] ")"  [ "," [ PtrSpec ] ( "(" [ PtrSpec ] Identifier AdtFuncPtrVarMember | Identifier AdtVarMember ) ] .
AdtBaseMember          = Identifier ( AdtMethodMember | AdtVarMember ) .
AdtMethodMember        = "(" [ AdtMethodRefParam [ "," ParamList ] ] | ParamList ")" .
AdtMethodRefParam      = ( "*" | "." ) Identifier [ Identifier ] .
AdtVarMember           = [ ArraySpec ] [ "," [ PtrSpec ] ( "(" [ PtrSpec ] Identifier AdtFuncPtrVarMember | Identifier AdtVarMember ) ] .
\end{lstlisting}

\hypertarget{enumerators}{%
\subsection{Enumerators}\label{enumerators}}

\begin{lstlisting}
EnumDecl   = "enum" [ Identifier ] "{" { EnumMember } "}" .
EnumMember = Identifier [ "=" Expression ]
\end{lstlisting}

\hypertarget{type-definitions}{%
\section{Type definitions}\label{type-definitions}}

Type definitions are declarations which start with the keyword
``typedef''. Type definitions can introduce new polymorphic variables in
the innermost active scope, forward references to complex types and new
names for basic and derived types.

\begin{lstlisting}
TypeDefs       = "typedef" ( PolyVarTypeDef | ForwardDef )
PolyVarTypeDef = BaseType [ PtrSpec ] ( DerivedTypeDef | FuncPtrTypeDef ) ";" .
DerivedTypeDef = Identifier [ ArraySpec ] .
FuncPtrTypeDef = "(" [ PtrSpec ] Identifier [ ArraySpec ] ")" "(" [ParamList] ")" .
ForwardDef     = ( "aggr" | "union" | "adt" ) Identifier ";" .
\end{lstlisting}

To declare complex types with mutually dependent pointers, it is
necessary to use a typedef to predefine one of the types. Alef does not
permit mutually dependent complex types, only references between them.
For example:

\begin{lstlisting}
typedef aggr A;
aggr B
{
    A *aptr;
    B *bptr;
};
aggr A
{
    A *aptr;
    B *bptr;
};
\end{lstlisting}

%\hypertarget{examples-of-type-definitions}{%
%\subsection{Examples of type
%definitions:}\label{examples-of-type-definitions}}
%
%\hypertarget{a-polymorphic-definition.}{%
%\subsubsection{A polymorphic
%definition.}\label{a-polymorphic-definition.}}
%
%\begin{lstlisting}
%typedef A;
%(TypeDefs => "typedef" PolyVarTypeDef => "typedef" Identifier ";" => "typedef" "A" ";")
%\end{lstlisting}
%
%\hypertarget{a-derived-type-definition.}{%
%\subsubsection{A derived type
%definition.}\label{a-derived-type-definition.}}
%
%\begin{lstlisting}
%typedef int * array_ptr_to_int [];  
%(TypeDefs => "typedef" PolyVarTypeDef => "typedef" Identifier DerivedTypeDef => 
%  => "typedef" Identifier PtrSpec Identifier ArraySpec => 
%    =*> "typedef" "int" "*" "array_ptr_to_int" "[]" ";")
%\end{lstlisting}
%
%\hypertarget{a-function-pointer-type-definition.}{%
%\subsubsection{A function pointer type
%definition.}\label{a-function-pointer-type-definition.}}
%
%\begin{lstlisting}
%typedef int (* array_ptr_to_func () ) (int, float, char) ";"
%\end{lstlisting}
%
