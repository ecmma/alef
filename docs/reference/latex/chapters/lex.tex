% !TEX root = ../main.tex 
\hypertarget{lexical-analysis}{%
\chapter{Lexical analysis}\label{lexical-analysis}}

Compilation starts with a preprocessing phase. An ANSI C preprocessor is
used, and any directive valid in such standard is valid in Alef as well.
The preprocessor performs file inclusion and macro substitution.
Comments and lines beginning with the character
\passthrough{\lstinline!\#!}are consumed by the preprocessor. The
preprocessor produces a sequence of tokens for the compiler\footnote{Note
  that using, for example, GCC's CPP on an Alef source file will produce
  a compilation unit with ``\#''-directives, which aren't described in
  this document, but is instead the objective of the compiler engineer
  to manage these implementation-dependent notions.}. Comments are
removed by the preprocessor. Any form of comment accepted by the
preprocessor is valid. For example, GNU's CPP might accept (and remove)
both \passthrough{\lstinline!/*...*/!} and
\passthrough{\lstinline!//..!} style comments.

\hypertarget{tokens}{%
\section{Tokens}\label{tokens}}

The lexical analyzer classifies tokens as identifiers, keywords,
literals and operators. Tokens are separated by white space, which is
ignored in the source except as needed to separate sequences of tokens
which would otherwise be ambiguous. The lexical analyzer is greedy: if
tokens have been consumed up to a given character, then the next token
will be the longest subsequent string of characters that forms a legal
token.

\hypertarget{reserved-words}{%
\subsection{Reserved words}\label{reserved-words}}

\hypertarget{keywords}{%
\subsubsection{Keywords}\label{keywords}}

The following symbols are keywords reserved by the language and may not
be used as user-defined identifiers:

\begin{lstlisting}
adt     aggr    alloc   alt     become    break   check   continue  default
do      else    enum    extern  float     for     goto    if        int
intern  lint    nil     par     proc      raise   rescue  return    sint
sizeof  switch  task    tuple   typedef   typeof  uint    ulint     unalloc
union   usint   void    while   zerox
\end{lstlisting}

Keywords that identify types, such as \passthrough{\lstinline!int!} or
\passthrough{\lstinline!chan!} are, from the point of view of lexical
analysis, seen as identifiers. It will be the job of the compiler to
make it so that at any time, if the parser has to distinguish between an actual
identifier and a typename, the symbol table at that point will
reflect the fact that \passthrough{\lstinline!int!} and
\passthrough{\lstinline!chan!}, for example, are predefined types. In the same
fashion, the \passthrough{\lstinline!nil!} keyword isn't interpreted as
a keyword, but as an always-defined constant of type
\passthrough{\lstinline!void\\*!} of value
\passthrough{\lstinline!0!}. In other words, to correctly build the parse tree
the parser needs the information to discern identifiers and typenames (otherwise, 
type declarations such as \passthrough{\lstinline!int* a!} would be undiscernible 
from multiplication expressions, as both would be seen as 
\passthrough{\lstinline!identifier op_mul identifier!}.)

\hypertarget{delimiters-and-operators}{%
\subsubsection{Delimiters and operators}\label{delimiters-and-operators}}

The following symbols are used as delimiters and operators in the
language:

\begin{lstlisting}
+   -   /   =   >   <   !   \%   &   |   ?   .   "   '   {   } 
[   ]   (   )   \*   ;   :   ^   +=  -=  /=  \*=  \%=  &=  |=  ^=
<<= >>= ==  !=  --  <-  ->  ++  ::  :=
\end{lstlisting}

\hypertarget{identifiers}{%
\subsection{Identifiers}\label{identifiers}}

An identifier, also called a lexical name, is any sequence of
alpha-numeric characters and the underscore character
\passthrough{\lstinline!\\\_!}. An identifiers binds a name to a
semantic object such as a type, a function or a variable. Identifiers
starting with ``ALEF'' are reseved for use in the runtime system.

\begin{lstlisting}
ASCII      = "a" | "b" | ... | "z" | "A" | "B" | ... | "Z" .
Digits     = "0" | "1" | ... | "9" .
Identifier = ( "_" | ASCII ) { ( "_" | ASCII | Digits ) } .
\end{lstlisting}

Identifiers may define variables, types, functions, function prototypes
or enumerators. An identifier has associated a
\protect\hyperlink{scopes}{scope} and a \protect\hyperlink{storage-classes}{storage classes}.

\hypertarget{literals}{%
\subsection{Literals}\label{literals}}

Alef literals are integer and floating point numbers, characters,
strings and runestrings (UTF-8 strings.) There are five types of
constant:

\begin{lstlisting}
Literal       = StringLit | RunestringLit | CharLit | IntLit | FloatLit .
StringLit     = '"' { ASCII | Escapes } '"' .
Escapes       = "\" ("0" | "n" | "r" | "t" | "b" | "f" | "a" | "v" | "\" | `"` ) .
RunestringLit = @TODO
CharLit       = @TODO .
IntLit        = [ "0" [ ( HexLit | OctalLit ) ] ] | DecimalLit .
HexLit        = ("x" | "X") { HexDigits }.
HexDigits     = "0" ... "9" | "A" ... "F" | "a" ... "f" .
OctalLit      = { OctalDigits } .
OctalDigits   = "0" ... "7"
DecimalLit    = ( "1" | ... | "9" ) { Digits } .
FloatLit      = @TODO . 
\end{lstlisting}

Character literals have the type \passthrough{\lstinline!uint!} and can
hold UTF-8 code points. String literals have type
\passthrough{\lstinline!static array of byte!} and are NUL
(\passthrough{\lstinline!\\0!}) terminated (appended by the compiler);
therefore, the sizeof operator applied to a string yields the number of
bytes including the appended NUL. Rune string literals are sequences of
UTF-8 code points and have type
\passthrough{\lstinline!static array of uint!}, and are NUL
(\passthrough{\lstinline!U+0000!}) terminated (appended by the
compiler); therefore, the sizeof operator applied to a string yields the
number of runes in the runestring, in terms of
\passthrough{\lstinline!sizeof(uint)!} including the appended NUL. The
following table shows valid characters after an escape and the value of
the constant:

\begin{lstlisting}
0     NUL     Null character
n     NL      Newline
r     CR      Carriage return
t     HT      Horizontal tab
b     BS      Backspace
f     FF      Form feed
a     BEL     Beep
v     VT      Vertical tab
\         \   Backslash
"     "   Double quote
\end{lstlisting}

Float literals have type \passthrough{\lstinline!float!}. Integer
literal have type \passthrough{\lstinline!int!}.

