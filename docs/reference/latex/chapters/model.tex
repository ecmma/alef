% TEX root = ../main.tex
\chapter{Execution model}

Alef is a concurrent programming language designed for
systems software. Exception handling, process management, and synchronization
primitives are implemented by the language. Expressions use the same syntax
as C, but the type system is substantially different. Alef supports object-oriented
programming through static inheritance and information hiding. The language does not
provide garbage collection, so programs are expected to manage their own memory.
This manual provides a description of the syntax and semantics of the
implementation.

\noindent
In the Alef execution model the term \textit{process} refers to a preemptively 
scheduled thread of execution. A process may contain severl \textit{tasks}, which 
are non-preemptively scheduled coroutines within a process. 

The memory model does not define the sharing of memory between processes; 
on a shared memory computer, processes will typically share the same address 
space, while on a multicomputer processes may be located on physically distant 
nodes with access only to local memory. In such a system, processes would not 
share the same address space, therefore must communicate using message passing. 
A group of tasks executing within the context of a process are defined to be 
in the same address space. Tasks are scheduled during communication (sending or 
receiving on/from channels) and synchronization (lock, unlock) operations. 
The term \textit{thread} is used wherever the distinction between a process and 
a task is unimportant. 

\textit{
In the current implementation, processes are implemented as plain OS forks, 
while tasks are implemented as green threads, using a round
robin scheduler implemented in the runtime -- scheduling of tasks is realized 
through simple calls to \texttt{\_alef\_yield}.}
